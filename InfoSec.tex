%%%%%%%%%%%%%%%%%%%%%%%%%%%%%%%%%%%%%%%%%%%%%%%%%%%%%%%%%%%%%%%%%%%%%%%%%%%%%%%%
%%  This is the main document for the COS 432 (Information Security) notes    %%
%%  from  Fall 2012. To compile these notes, you should put the following     %%
%%  files in the same directory:                                              %%
%%    InfoSec.tex (this file), InfoSecPreamble.tex, InfoSecTitle.tex,         %%
%%    Lecture01.tex, Lecture02.tex, ...                                       %%
%%  Then run 'latex InfoSec.tex' (without quotes) three times (twice might be %%
%%  enough).                                                                  %%
%%                                                                            %%
%%  If you want to get a pdf, don't use pdflatex, because I use pstricks.     %%
%%  Instead, use latex, then dvips, then ps2pdf.                              %%
%%                                                                            %%
%%  Brenda (all format copied from Anton:                                     %%
%%      http://stacky.net/wiki/index.php?title=Course_notes)                  %%
%%%%%%%%%%%%%%%%%%%%%%%%%%%%%%%%%%%%%%%%%%%%%%%%%%%%%%%%%%%%%%%%%%%%%%%%%%%%%%%%

%% The preamble loads packages, theorem styles, and macros %%%%%%%%%%%%%%%%%%%%%
\documentclass[12pt,twoside]{article}

 \usepackage{amsmath}   % Some symbols
 \usepackage{amsthm}    % Does theorem stuff
 \usepackage{amssymb}       % More symbols and fonts
 \usepackage{empheq}        % Some more extensible arrows, like \xmapsto
 \usepackage{enumerate}     % Formatting of enumerates
 \usepackage{fancybox}      % More outline options for boxes
 \usepackage{mathrsfs}      % Sheafy font \mathscr{}
 \usepackage{multirow}      % For tables
 
 %% (this package is not needed for the 2015-01-19 version of this manuscript
 %% and beyond)
 %%
 %% \usepackage{pstricks}      % PSTricks!
 
 \usepackage{rotating}      % Rotate text
 \usepackage{xcolor}        % For colors (links)
 \usepackage[all]{xy}       % Include XY-pic
    \SelectTips{cm}{10}     % Use the nicer arrowheads
    \everyxy={<2.5em,0em>:} % Sets the scale I like
 \usepackage[colorlinks,
             linkcolor=black,
             pagebackref,
             bookmarksnumbered=true]{hyperref}

%% Pagestyle stuff %%%%%%%%%%%%%%%%%%%%%%%%%%%%%%%%%%%%%%%%%%%%%%%%%%%%%%%%%%%%%
 % Begin paragraphs with an empty line rather than an indent                  %%
 \usepackage[parfill]{parskip}                                                %%
 \usepackage{fancyhdr}                                                        %%
   \pagestyle{fancy}                                                          %%
   \fancyhf{}   % Delete the current section for header and footer            %%
 \usepackage[paperheight=11in,                                                %%
             paperwidth=8.5in,                                                %%
             outer=1.2in,                                                     %%
             inner=1.2in,                                                     %%
             bottom=.7in,                                                     %%
             top=.7in,                                                        %%
             includeheadfoot]{geometry}                                       %%
   \addtolength{\headwidth}{.75in}                                            %%
   \fancyhead[RO,LE]{\thepage}                                                %%
   \fancyhead[RE,LO]{\sectionname}                                            %%
   \setlength{\headheight}{15.8pt}                                            %%
   \raggedbottom                                                              %%
%% End Pagestyle stuff %%%%%%%%%%%%%%%%%%%%%%%%%%%%%%%%%%%%%%%%%%%%%%%%%%%%%%%%%

%% Stuff for keeping track of sections %%%%%%%%%%%%%%%%%%%%%%%%%%%%%%%%%%%%%%%%%
 \newcommand{\sektion}[2]{\stepcounter{section}                               %%
     \renewcommand{\thesection}{#1}                                           %%
     \newpage\section{#2} \gdef\sectionname{#1\quad #2}}                      %%
 \newcommand{\subsektion}[1]{\subsection*{#1}                                 %%
     \addcontentsline{toc}{subsection}{#1}}                                   %%
 % This is the empty section title, before any section title is set           %%
 \newcommand\sectionname{}                                                    %%
%% End stuff for keeping track of sections %%%%%%%%%%%%%%%%%%%%%%%%%%%%%%%%%%%%%

%% Theorem Styles and Counters %%%%%%%%%%%%%%%%%%%%%%%%%%%%%%%%%%%%%%%%%%%%%%%%%
 \renewcommand{\theequation}{\thesection.\arabic{equation}}                   %%
 \makeatletter                                                                %%
    % Make the equation counter reset each section                            %%
    \@addtoreset{equation}{section}                                           %%
    % Make the footnote counter reset each section                            %%
    \@addtoreset{footnote}{section}                                           %%
                                                                              %%
 \newenvironment{warning}[1][]{                                               %%
    \begin{trivlist} \item[] \noindent                                        %%
    \begingroup\hangindent=2pc\hangafter=-2                                   %%
    \clubpenalty=10000%                                                       %%
    \hbox to0pt{\hskip-\hangindent\manfntsymbol{127}\hfill}\ignorespaces      %%
    \refstepcounter{equation}\textbf{Warning~\theequation}                    %%
    \@ifnotempty{#1}{\the\thm@notefont \ (#1)}\textbf{.}                      %%
    \let\p@@r=\par \def\p@r{\p@@r \hangindent=0pc} \let\par=\p@r}             %%
    {\hspace*{\fill}$\lrcorner$\endgraf\endgroup\end{trivlist}}               %%
                                                                              %%
 \newenvironment{exercise}[1][]{\begin{trivlist}                              %%
    \item{\bf Exercise\@ifnotempty{#1}{ #1}.}\it}{\end{trivlist}}             %%
 \newenvironment{solution}{\begin{trivlist}                                   %%
    \item{\it Solution.}}{\end{trivlist}}                                     %%
                                                                              %%
 \def\newprooflikeenvironment#1#2#3#4{                                        %%
      \newenvironment{#1}[1][]{                                               %%
%          \refstepcounter{equation}                                          %%
          \begin{proof}[{\rm\csname#4\endcsname{#2}\@ifnotempty{##1}          %%
              {\the\thm@notefont\(##1)}\csname#4\endcsname{.}}]               %%
          \def\qedsymbol{#3}}                                                 %%
         {\end{proof}}}                                                       %%
 \makeatother                                                                 %%
                                                                              %%
 \newprooflikeenvironment{definition}{Definition}{$\diamond$}{textbf}         %%
 \newprooflikeenvironment{example}{Example}{$\diamond$}{textbf}               %%
 \newprooflikeenvironment{remark}{Remark}{$\diamond$}{textbf}                 %%
                                                                              %%
 \theoremstyle{plain}                                                         %%
 \newtheorem{theorem}[equation]{Theorem}                                      %%
 \newtheorem*{claim}{Claim}                                                   %%
 \newtheorem*{lemma*}{Lemma}                                                  %%
 \newtheorem*{theorem*}{Theorem}                                              %%
 \newtheorem{lemma}[equation]{Lemma}                                          %%
 \newtheorem{corollary}[equation]{Corollary}                                  %%
 \newtheorem{proposition}[equation]{Proposition}                              %%
%% End Theorem Styles and Counters %%%%%%%%%%%%%%%%%%%%%%%%%%%%%%%%%%%%%%%%%%%%%

%% Misc %%%%%%%%%%%%%%%%%%%%%%%%%%%%%%%%%%%%%%%%%%%%%%%%%%%%%%%%%%%%%%%%%%%%%%%%
 \newenvironment{sidenote}[1]{
    \shadowbox{\parbox{\linewidth}{#1}}
 }
%% End Misc %%%%%%%%%%%%%%%%%%%%%%%%%%%%%%%%%%%%%%%%%%%%%%%%%%%%%%%%%%%%%%%%%%%%

%% Macros %%%%%%%%%%%%%%%%%%%%%%%%%%%%%%%%%%%%%%%%%%%%%%%%%%%%%%%%%%%%%%%%%%%%%%
 \newcommand{\brenda}[1]{[[\ensuremath{\bigstar\bigstar\bigstar} #1]]}        %%
 \newcommand*\xor{\oplus}
%% End Macros %%%%%%%%%%%%%%%%%%%%%%%%%%%%%%%%%%%%%%%%%%%%%%%%%%%%%%%%%%%%%%%%%%


\begin{document}{

% Any subset of the following lines can be commented out

 \title{\vspace*{-2cm} Notes for COS 432 - Information Security\vspace*{-12mm}\footnote{This work is licensed under
a Creative Commons Attribution-NonCommercial 4.0 International license.  
For details see \url{https://creativecommons.org/licenses/by-nc/4.0/}}}
 \author{}
 \date{}
 \maketitle
 \phantomsection    % This makes the hyperref package happier for some reason

 {\thispagestyle{empty}
  \vspace*{-2em}
%  \addcontentsline{toc}{section}{Contents}
  \tableofcontents
 }
}{  % Title page, Contents

%!TEX root = InfoSec.tex 
% Lecture 1: 10 September 2014 with Ed Felten

\sektion{1}{Message Integrity}
\subsektion{Sending messages}
\ovalbox{Alice} $\xrightarrow{m}$
    \ovalbox{Mallory} $\xrightarrow{?}$
    \ovalbox{Bob}

{\bf Threat Models:}, what adversary can do and accomplish vs. what we want to do and accomplish. We generally assume
         that Mallory is malicious in the most devious possible way, as opposed
         to random error. In this case of Alice sending Bob a message: 
\begin{itemize}
    \item Mallory can see and forge messages
    \item Mallory wants to get Bob to accept a message that Alice didn't send
    \item Alice and Bob want Alice to be able to send a message and have Bob receive it in an untampered form.
\end{itemize}

\sidenote{
    {\bf CIA Properties}
    \begin{itemize}
        \item Confidentiality: trying to keep information secret from someone
        \item Integrity: making sure information hasn't been tampered with
        \item Availability: making sure system is there and running when needed (hardest to achieve!)
    \end{itemize}
}

In this problem, the goal is only integrity. \\

\sidenote{
    {\bf Role of stories in security:}
    \begin{itemize}
        \item Pro: easy to follow
        \item Cons:
        \begin{itemize}
            \item In reality,``Alice/Bob'' is a computer; for example, a server with no common sense
            \item In reality, ``Alice/Bob'' is a person + computer (one may have some
                    knowledge that other doesn't, e.g. knowledge divergence in
                    phishing attack)
            \item We might be biased into rooting for one side or the other and lose impartiality
        \end{itemize}
    \end{itemize}
}

What to send:\\

\ovalbox{Alice} $\xrightarrow{(m, f(m))}$
    \ovalbox{Mallory} $\xrightarrow{(a,b)}$
    \ovalbox{Bob} : accept $a$ iff $f(a) = b$

where $f$ is a {\bf Message Authentication Code (MAC)}\\

Properties $f$ needs to be a secure MAC:
\begin{enumerate}
    \item deterministic (Bob needs to get the same answer that Alice got every time)
    \item easily computable by Alice and Bob
    \item not computable by Mallory (else Mallory can send $(x, f(x))$ for any
        $x$ s/he wants)
\end{enumerate}

Choosing $f$:
\begin{itemize}
    \item Picking a secret function is risky because it is difficult 
    	to quantify how likely Mallory will be able to guess the function.
    \item Use a random function...
        \begin{table}[!h]\centering\begin{tabular}{r|ll}
            input & output &\\
            \cline{1-2}
            $\emptyset$ & 01011... & $\leftarrow$ 256 coin flips\\
            0 & 101... &\\
            1 & ... &\\
        \end{tabular}\end{table}

\sidenote{
    {\bf ``secure MAC game'': Us vs. Mallory}

    \hspace*{0.5 cm} repeat until Mallory says ``stop'': \{\\
        \hspace*{1 cm} Mallory chooses $x_i$\\
        \hspace*{1 cm} we announce $f(x_i)$\\
    \hspace*{0.5 cm} \}\\
    \hspace*{0.5 cm} Mallory chooses $y \not\in \{x_i\}$\\
    \hspace*{0.5 cm} Mallory guesses $f(y)$: wins if right
    \\

    $f$ is a secure MAC if and only if every efficient (polytime) strategy for Mallory 
    wins with negligible (probability that goes to 0) probability. In other words, 
    $f$ is a secure MAC if Mallory can't do better than random guessing.
    
    \bigskip
    
    \begin{theorem*}{A random function is a secure MAC.}\end{theorem*}
    \emph{Intuition:} Mallory asks to reveal certain entries, but for $y$
        Mallory is trying to guess the result of the coin flips
}
    \item ...Or more practically, a pseudorandom function:

        {\bf pseudorandom function (PRF)}: ``looks random'', ``as good as
            random'', practical to implement

        typical approach:
        \begin{itemize}
            \item \underline{public} family of function $f_0, f_1, f_2, \dots$
            \item \underline{secret} key $k$ which is, for example, a 256 bit random value
            \item \underline{use} $f(k, x)$
        \end{itemize}
        
\sidenote{
    {\bf Kerckhoffs's principle:}

    Use a public function family and a randomly chosen secret key.
    \bigskip
    Advantages:
    \begin{enumerate}
        \item can quantify probability that key will be guessed
        \item different people can use the same functions with different keys
        \item can change key if needed (if it's given out or lost)
    \end{enumerate}
}
\\
\sidenote{
    {\bf ``PRF game'' against Mallory}:

    \hspace*{0.5 cm} we flip a coin secretly to get $b \in \{0,1\}$\\
    \hspace*{0.5 cm} if $b = 0$, let $g = $ random function\\
    \hspace*{0.5 cm} else, $g=f(k, x)$ for random $k$\\
    \hspace*{0.5 cm} repeat until Mallory says ``stop'': \{\\
        \hspace*{1 cm} Mallory chooses $x_i$\\
        \hspace*{1 cm} we announce $g(x_i)$\\
    \hspace*{0.5 cm} \}\\
    \hspace*{0.5 cm} Mallory guesses latest $b$: wins if right\\
    
    
    $f$ is a PRF if and only if every efficient strategy for Mallory wins with probability less than
    $0.5 + \epsilon$ where $\epsilon$ is negligible. \\

    Note: Mallory can always win by exhaustive search of the range of $k$ in $f(k, x)$, so need to
    limit Mallory to ``practical''\\

    \begin{theorem*}{If $f$ is a PRF, then $f$ is a secure MAC}\end{theorem*}
    \begin{proof} By contradiction. There's a reduction going on; we wanted
    to find a secure MAC, which led us to wanting to find a secure PRF
    \end{proof}
}
\end{itemize}

What to send (new):\\

\ovalbox{Alice} $\xrightarrow{(m, f(k, m)}$
    \ovalbox{Mallory} $\xrightarrow{(a,b)}$
    \ovalbox{Bob} : accept $a$ iff $f(k, a) = b$

Assumptions:
\begin{enumerate}
    \item $k$ is kept secret from Mallory
    \item Alice and Bob have established $k$ in advance
    \item Mallory doesn't tamper with the code that computes the function $f(k, a)$
\end{enumerate}

\subsektion{Do PRF's exist?}
Answer: maybe/ we hope so (some functions haven't lost yet)\\

Here's one: HMAC-SHA256
$$f(k, x) = S((k \xor z_1) || S((k \xor z_2) || x))$$
where $z_1 = 0x3636\dots$, $z_2 = 0x5c5c\dots$ (note that $||$ is concatenation)
and $S$ is ``SHA-256'': start with ``compression function'' $C$, taking 256 and
512 bits in, outputting 256 bits

\makebox[2cm]{}\framebox[8cm]{input}\framebox[2cm]{pad}\\
\makebox[2cm]{}\framebox[2cm]{}\framebox[2cm]{}\framebox[2cm]{}\framebox[2cm]{}
\mbox{512 bit blocks}\\
\makebox[2cm]{}\makebox[2cm]{$\Downarrow$}\makebox[2cm]{$\Downarrow$}
\makebox[2cm]{$\Downarrow$}\makebox[2cm]{$\Downarrow$}\\
\makebox[2cm]{const $\rightarrow$}\makebox[2cm]{C}\makebox[.2cm]{$\rightarrow$}
\makebox[1.4cm]{C}\makebox[4.4cm]{$\cdots$}\makebox[2cm]{output}

Note: This is subject to length extension attacks\\

\subsektion{Cryptographic Hash Functions}
They include MD5, SHA-1, SHA-?, etc: functions that take arbitrary size inputs and return
fixed size outputs that are ``hard to reverse." They are dangerous to use directly because
they don't have the properties you think/want then to have.\\

Properties of a cryptographic hash function
\begin{enumerate}
	\item Collision resistance:\\
		Can't find $x \neq y$ such that $H(x) = (y)$
	\item Second preimage resistance:\\
		Given $x$, can't find $y$ such that $H(x) = H(y)$
	\item If $x$ is chosen randomly from a distribution \textit{with high entropy},
	then given $H(x)$, you can't find $x$
\end{enumerate}

Better: use a PRF even if $k$ is non-secret

\subsektion{Timing Attacks}
Suppose Alice and Bob implement MAC-based integrity with the following code

\begin{verbatim}
def macCheck(a, b, key) {
    correctMac = Mac(key, a);
    for (i = 0; i < length; ++i) {
        if (correctMac[i] != b[i]) return false
        }
    return true
}
\end{verbatim}

The problem? The execution time depends on the first $n$ correct characters. Mallory may observe the runtime to
gain insight on cracking the code.

\subsektion{Multiple Alice - Bob messages}
How to deal with Mallory sending messages out of order or resending old messages
\begin{enumerate}
\item append sequence number to each message:\\
        Alice sends $m_0' = (0, m_0)$, $m_1' = (1, m_1)$
\item switch keys per message
\end{enumerate}
}{
%!TEX root = InfoSec.tex
% Lecture 2: 19 September 2012
\sektion{2}{Randomness}
Best way to get a value that is unknown to an adversary is to choose a random
value, but it's hard to get this in practice. Randomness (or a lack thereof) is often a weakness in a security system.\\

Recall from last lecture that a PRF works as a MAC.

\sidenote{
    {\bf What is a PRF?}\\

    Two views:
    \begin{enumerate}
        \item family of functions $f_k(x)$
        \item function $f(k,x)$. This is the view we'll be using for this class.
    \end{enumerate}
}

{\bf True randomness}:
\begin{itemize}
    \item outcome of some inherently random process
    \item assume it ``exists'' but it's scarce and hard to get
\end{itemize}

{\bf In security, "random" means unpredictable}:
\begin{itemize}
    \item to whom? E.x., in a PRF, the result can be considered random with respect to someone who does not know the secret key.
    \item when?
\end{itemize}

{\bf Pseudorandom generator (PRG)}:
\begin{itemize}
    \item takes a small ``seed'' that's truly random as input E.x. a few coin flips, instead of flipping a coin each time
    \item generates a long sequence of ``good enough'' values, i.e. unlimited pseudorandomness.
    \item maintains ``hidden state'' that changes as generator operates
    \item output is indistinguishable from truly random output in the practical sense, i.e. an efficient party can not distinguishs
    \item the generator needs to be deterministic, because if it is not it must be driven by some kind of randomness, and the reason we are doing this is because randomness is scarce
\end{itemize}

{\bf Randomness service}:
\begin{itemize}
    \item OS service, callable by application
    
\end{itemize}

\begin{definition}
PRG is {\bf secure} if its output is indistinguishable from a truly random
value/string.\\ This is based on the game versus Mallory (can Mallory
tell real randomness from prg? similar to prf game from lecture 1),
where secure means that Mallory wins 50\% ($+\epsilon$) assuming
Mallory has limited resources, where $\epsilon$ is negligible.
\end{definition}

\begin{tabular}{cccccccc}
&&&&&&& \\
Seed & $\xrightarrow{init}$ & $ S_0 $ & $\xrightarrow{advance}$ & $ S_1$ & $\xrightarrow{advance}$ & $S_2$ & etc.\\
& & $\downarrow$ & & $\downarrow$ & &  $\downarrow$ &\\
& & $output_0$ & & $output_1$ & &  $output_2$ &\\
&&&&&&&
\end{tabular}

Another desirable property is {\bf Forward Secrecy} (backtracking resistance):\\
If Mallory compromises the hidden state of the generator at time $t$, Mallory
can't backtrack to reconstruct past outputs of the generator.\\

Note that if an adversary breaks in at time $t$, they \textit{can} play it forward and see the outputs at time $t + x$

Most PRGs are made up of an \textbf{init} function to initialize state
$S$ and an \textbf{advance} function to step to a new state.

\begin{example}{A PRG that is \underline{not} FS but is secure:}
    \begin{itemize}
    \item Let $f$ be a PRF
    \item init: $(seed, 0)$
    \item advance: $(seed, k) \rightarrow (seed, k + 1)$
    \item output: $f(seed, k)$
    \end{itemize}
    If Mallory knows the counter $k$ at any point, she can decrement it and run the function forwards again. 
\end{example}

\begin{example}{A PRG that is FS and secure:}
    \begin{itemize}
    \item Let $f$ be a PRF
    \item init: $seed$
    \item advance: $S \rightarrow f(S, 0)$
    \item output: $f(S, 1)$
    \end{itemize}
    This resists backtracking because the advance function relies on the PRF, and the seed is overwritten
\end{example}

\subsektion{Randomness as a system service}
Hard parts: getting seed, recovering from compromise, even if we don't know whether the state has been compromised. We want to be continuously recovering because we might not notice a compromise.\\
Create a new function, $recover(S, random data) \rightarrow state$\\

Getting a good seed: want true randomness
\begin{itemize}
    \item special circuit
    \item ambient audio/video: lava lamps! (lavarand)
\end{itemize}
problems: physical random processes are difficult in practice, not \emph{truly} random (correlations)\\

Alternate view: \textbf{collect} data unpredictable to adversary
\begin{itemize}
    \item exact history of key presses
    \item exact path of mouse
    \item exact history of packet traffic
    \item periodic screenshot
    \item internal temperature
    \item ambient audio
\end{itemize}


Then: process to \textbf{extract}, or distill down to ``pure''
randomness - feed it all into a PRF. If there's enough randomness in
input, output will be ``pure random''.  Can, for example, use
SHA256(all the data). SHA256 consumes data one block at a time, so we don't 
need to collect and store all the data; we can get/use the data iteratively.\\

 Use this to:
\begin{itemize}
    \item seed the system PRG
    \item recover/renew the state (mix fresh randomness in with hidden state) using PRF,
        to re-establish secrecy of hidden state\\

        NOTE: Mistake to add a single bit at a time since Mallory can keep
        up with 2 possibilities at a time, but if we wait until have a
        lot, say 256 bits of randomness, then Mallory can't keep up ($2^{256}$
        possibilities), even if she knows the algorithm used.
\end{itemize}

Hard to estimate actual amount of entropy in pool, so wait for too
much randomness before mixing to remain conservative.\\
There's also a problem with ``headless'' machines, like servers, that don't have
enough areas of randomness to draw from.

{\bf Linux}:
\begin{itemize}
    \item {\tt /dev/random} gives pure random bits, but have to wait
    \item {\tt /dev/urandom} is output of PRG, renewed via ``pure'' randomness
\end{itemize}

The boot problem: At startup,
\begin{itemize}
    \item least access to randomness (system is clean)
    \item highest demand for randomness (programs want keys)
\end{itemize}

Solutions (with their problems):
\begin{itemize}
    \item save some randomness only accessible at boot:\\
        hard to tell that this hasn't been observed, or used on last boot
    \item connect to someone across network to give pseudorandomness:\\
        want secure connection but don't yet have key (okay if have just enough
        for that key, or semi-predictable and hope Mallory doesn't guess)
\end{itemize}

\subsektion{Message Confidentiality}
Now may have a (passive) adversary/eavesdropper Eve who can only listen:\\
\makebox[5cm]{\ovalbox{Alice} $\rightarrow$ \ovalbox{Bob}}\\
\makebox[5cm]{$\downarrow$}\\
\makebox[5cm]{\ovalbox{Eve}}\\

Message processing:\\
\makebox[1.5cm]{$\xrightarrow{\text{plaintext}}$}
    \framebox[2.5cm]{$E$ (encrypts)}
    \makebox[1.5cm]{$\xrightarrow{\text{ciphertext}}$}
    \framebox[2.5cm]{$D$ (decrypts)}
    \makebox[1.5cm]{$\xrightarrow{\text{plaintext}}$}\\
\makebox[1.5cm]{}\makebox[2.5cm]{$\uparrow$}
    \makebox[1.5cm]{}\makebox[2.5cm]{$\uparrow$}\\
\makebox[1.5cm]{}\makebox[2.5cm]{key $k$}
    \makebox[1.5cm]{}\makebox[2.5cm]{key $k$}\\

Goal: ciphertext does not convey anything about plaintext. Bob can recover text, Eve cannot.

\textbf{Semantic Security}

\sidenote{
    {\bf ``Encryption game'' against Eve:}\\\\
    \hspace*{0.5cm} Allow Eve to pick piece of plaintext, we provide encryption $E_k(x_i)$ until she is satisified
    \hspace*{0.5 cm} Eve chooses two pieces of plaintext\\
    \hspace*{0.5cm} We flip a coin and encrypt one of them\\
    \hspace*{0.5cm} Eve guesses which was encrypted: wins if right

    We say that the encrpytion method is  secure if Eve can't do
    better than random guessing (50/50) + negligible $\epsilon$.  This is known as
    \textbf{semantic security}.\\

    Note: if we were being more rigorous in our definitions, we would use a
    stronger definition of security for encryption here so that it's easier to
    combine later with integrity. However, the methods we are learning are
    secure by any of the definitions.
}

{\bf First approach: one-time pad (known to be semantically secure)}
\begin{enumerate}
    \item Alice and Bob jointly generate a long random string $k$ (``the pad'')
    \item $E(k, x) = k \xor x$
    \item $D(k, y) = k \xor y = k \xor (k \xor x) = (k \xor k) \xor x = x$
\end{enumerate}
Problems:
\begin{enumerate}
    \item can't reuse key:\\
        $(k \xor a) \xor (k \xor b) = a \xor b$\\
        worst case, Eve knows one message, but even knowing that the messages
        are say English text can give Eve information from character
        distributions
    \item need really long key -- needs to be as long as sum of message lengths
\end{enumerate}
Idea: use a PRG to ``stretch'' a small key (called a ``stream cipher'')
\begin{itemize}
    \item Start with fixed-size random $k$, add a ``nonce'': unique, i.e. don't resue nonce value,
      but not secret.  Use PRF(k, nonce) to seed a PRG.
    \item Alice and Bob run identical PRGs in parallel with same key
    \item xor messages with PRG's output
    \item Do not re-use (key, nonce) pair
\end{itemize}
  This approach still does not provide integrity.

\subsektion{Confidentiality and integrity}
Few approaches.
\begin{enumerate}
  \item Use E(x $||$ M(x)) \hspace{1cm}SSL/TLS
  \item Use E(x) $||$ M(E(x)) \hspace{1cm}IPSec  **This is the winner (because math).
  \item Use E(x) $||$ M(x) \hspace{1cm}SSH\\
\end{enumerate}

\begin{theorem}
If E is a semantically secure cipher, and M is a secure MAC, then \#2 is secure.
\end{theorem}

Encrypt plaintext, then append MAC: Bob first integrity checks, then decrypts.
Note that we need to use separate keys for confidentiality and integrity, and a
separate set of two keys for reverse channel (Bob to Alice).\\

If we have only one shared key, we seed the PRG with the shared key and then use
four  values it produces for the message sending.
}{
%!TEX root = InfoSec.tex
% Lecture 3: 17 September 2014
\sektion{3}{Block ciphers}

\sidenote{
    {\bf Story from WWII:}

    Pacific war: lots of radio communications; crypto and US decryptions paid a
        huge role

    Admiral Nimitz had advantage of code break giving Japanese battle plan
        (Battle of Midway)

    Most successful code was used by the US Marines: the Navajo language served
        as a code by translating the first letter of an English word into a
        Navajo word and sending that by radio (allowed speech communication).

    Even when they got a Navajo speaker, the Japanese were unable to usefully
        decrypt these messages.
}

Last time: Stream ciphers
    $$E(k,m) = D(k, m) = \text{PRG}(k) \xor m$$

Alternative approach: Block ciphers

Start with function that encrypts a fixed-size block of data (and fixed-size
        key) and build up from there
\begin{itemize}
    \item may run faster
    \item many PRGs work this way anyway
\end{itemize}

Note that block cipher is not the same thing as a PRF, since a PRF may have no
    inverse ($\exists x_1, x_2 \text{ s.t. } f(k,x_1) = f(k,x_2)$

Want: psuedorandom permutation (PRP)
\begin{itemize}
    \item function from $n$-bit input (plus key) to $n$-bit output
    \item if $x_1 \neq x_2$, then $f(k, x_1) \neq f(k, x_2)$
    \item psuedorandom as expected from previous definitions -- should be
        indistinguishable from truly random to an adversary
\end{itemize}

It is useful to compare the different types of security functions we have
seen. Note: Can use any of PR function/permutation/generator to build
the other two.

\begin{center}
\begin{tabular}{l|llll}
 Property    &  PR Functions         &  PR Permutations  &  PR Generators  &  Hash Functions       \\
\hline
 Input       &  Any                  &  Fixed-size       &  Fixed-size     &  Any                  \\
 Output      &  Fixed-size           &  Fixed-size       &  Any            &  Fixed-size           \\
 Has Key     &  Yes                  &  Yes              &  Yes            &  No                   \\
 Invertible  &  No                   &  With key         &  No             &  Depends              \\
 Collisions  &  Yes, but can't find  &  No               &  No             &  Yes, but can't find  \\
\end{tabular}
\end{center}

Challenge: design a very hairy function that's invertible, but only by someone
who knows the key. A PRP will have this property.\\

Minimal properties of a good block cipher:
\begin{itemize}
    \item efficient
    \item highly nonlinear (``confusion property'') - hard for adversary to
        invert
    \item mix input bits together (``diffusion'') - every input bit affects the output
    \item depend on the key
\end{itemize}

How to get these properties: Feistel network, given that $f$ is a PRF

\makebox[6cm]{plaintext}\\
\framebox[6cm]{\makebox[3cm]{left half}\textbar\makebox[3cm]{right half}}\\
\makebox[6cm]{\makebox[1cm]{$\downarrow$}\makebox[4cm]{}\makebox[1cm]{$\downarrow$}}\\
\makebox[6cm]{\makebox[1cm]{$\xor$}\makebox[1cm]{$\longleftarrow$}\framebox[2cm]{$f(k_0)$}\makebox[1cm]{$\longleftarrow$}\makebox[1cm]{$\downarrow$}}   ``feistal round''\\
\makebox[6cm]{\makebox[1cm]{$\downarrow$}\makebox[4cm]{}\makebox[1cm]{$\downarrow$}}\\
\makebox[6cm]{------------------------------------------------}\\
\makebox[6cm]{\makebox[1cm]{$\downarrow$}\makebox[1cm]{$\longrightarrow$}\framebox[2cm]{$f(k_1)$}\makebox[1cm]{$\longrightarrow$}\makebox[1cm]{$\xor$}}   another round\\
\makebox[6cm]{\makebox[1cm]{$\downarrow$}\makebox[4cm]{}\makebox[1cm]{$\downarrow$}}\\

Add as many rounds as you want, alternating left and right rounds.\\

Why? Easy to invert since each round is its own inverse, so inverse of series
of rounds is the same series in reverse order. This makes it so that $f$ can be
as difficult as we want and the process is still invertible, so why not make $f$
a PRF?\\

\begin{theorem*}
If $f$ is a PRF, then 4-round feistel network is a PRP.
\end{theorem*}
Often use weaker rounds (which may be faster to compute), but more of them.

\begin{example}{DES (Data Encryption Standard)}
    \begin{itemize}
        \item block size = 64 bits
        \item 56-bit key
        \item Feistel network - 16 (weak) rounds
    \end{itemize}

    History:
    \begin{itemize}
        \item designed in secrey by IBM and NSA in 1978
        \item US government standard
        \item private sector followed
    \end{itemize}

    Backdoor known to designers but not public? Concerns by public over the
    source - history of US offering other governments intentionally weak ciphers

    Reason for secrecy around design was that some of the classification: wanted
    to make secure against differential cryptanalysis, but that wasn't publicly
    known yet and NSA wanted to keep using it against others.

    Designed to be slow in software to discourage it from being implemented in
    software

    Key size problem:\\
    $2^{56}$ steps for a brute-force search to recover an unknown key, which is
    currently feasable, though not in 1978 (except maybe by NSA?)

    This can be addressed by iterating DES with multiple keys. Note that you
    need to do this three times to get $2^{112}$ for brute force search.
\end{example}

\begin{example}{AES (Advanced Encryption Standard)}
    Probably the best available today, coming from overcoming drawbacks of DES
    \begin{itemize}
        \item software efficiency a goal
        \item large, variable key size (128-, 192-, 256-bit variants)
        \item open, public process for choosing and generating the cipher\\
            run by NIST and a design contest judged on pre-determined criteria
    \end{itemize}

    2002 - NIST chose Rijndael (Belgian designers)
\end{example}

\subsektion{128-bit AES}
\begin{itemize}
    \item 128-bit input, output, and key
    \item not feistel design
    \item lookup table public

    \item ten rounds (generally cryptanalysis is a small-round break and then
            extending the tactic to a full number of rounds, so use a safe
            number then add extra rounds for safety buffer), each with four steps
        \begin{enumerate}
            \item non-linear step (``confusion''):

                run each byte through a certain non-linear function (lookup
                table of a permutation)
            \item shift step (``diffusion''):

                think of 128-bits as 4x4 array of bytes: shift the $i$th row left $i$ steps; values that 
                fall off wrap around the same row. (circular shift)

                spreads out columns
            \item linear mix (``diffusion''):

                take each column, treat as 4-vector and multiply by a certain
                matrix (specified in standard)

                mixes within column
            \item key-addition step (key-dependent):

                xor each byte with corresponding byte of the key

                Note: the key expansion could be a source of weakness (to get
                the ten keys needed from one)
        \end{enumerate}
    \item to decrypt, do inverses in reverse order
\end{itemize}

\subsektion{How to handle variable-size messages}
Problems:
\begin{itemize}
    \item padding - plaintext not a multiple of blocksize
    \item ``cipher modes'' - dealing with multi-block messages
\end{itemize}

Padding: most important property needs to be that recipient can unambiguously
    tell what is padding and what is not\\

    Good method: add bits 10* until reach end of block (pull off all 0's at end
    then the 1). Remember that you must add {\emph some} padding (at least one
    bit) to every message. This works similarly with bytes.\\

\bigskip
Cipher modes: encrypt multi-block messages
\begin{itemize}
    \item ECB (Electronic Code Book) !! BAD - not semantically secure - do not use !!
        $$C_i = E(k, P_i)$$
        Same plaintext results in same ciphertext -- leaks information to
        adversary
    \item CBC (Cipher Block Chaining): Common, pretty good\\
        ``strawman CBC'', $R_i$ random
        $$C_i = (R_i, E(k, R_i \xor P_i)$$
        Good, but doubles message size\\
        Idea: use $C_{i-1}$ instead of $R_i$\\
        $$C_i = E(k, C_{i-1} \xor P_i)$$
        What about the first block? Generate a random value, the
        ``initialization vector (IV)'', to prepend to message to serve as
        $C_{-1}$. Don't want to reuse with same key, or adversary could compare
        the first block of the ciphertext to see if same plaintext, but
        random-ish generation good enough, and can use same key over and over.
    \item CTR (Counter mode): Generally agreed on as best to use.
        Similar to a stream cipher.
        $$C_i = E(k, \text{messageid} || \text{counter}) \xor P_i$$
        messageid must be unique, then it's okay to reuse key.\\
        Note: this would not be forward secret as a PRG.\\
        Reasons to use CTR over PRG: more efficient on commodity hardware and
        perhaps you trust AES more than your PRF (even though you can't prove it
        either way).
\end{itemize}
}{
%!TEX root = InfoSec.tex
% Lecture 4: 22 September 2014
\sektion{4}{Asymmetric key cryptography}
Symmetric key: use the same key to encrypt and decrypt

Problems:
\begin{itemize}
    \item Integrity: Alice sending to Bob, Charlie, Diana, ...

        If Alice, Bob, Charlie, Diana all have key $k$, then Bob could compute
        a MAC on a message and deliever a message to Charlie and Diana, thereby
        forging a message

        It would be nice if Alice were the only one who could send a verified
        message without needing to append everyone's integrity key (one per
        recipient)
    \item Confidentiality: maybe only Alice should be able to decrypt a message
\end{itemize}

Asymmetric scheme: 1976, Diffie-Hellman(-Cox for British military)
\begin{itemize}
\item One key for encrypting, another for decrypting
\item One key for MAC, another for verifying it
\end{itemize}

\begin{definition}
{\bf ``public-key'' cryptography}

    Almost always:
    \begin{itemize}
        \item Generate key-pair such that can't derive one key from the other
        \item One key is kept private (only Alice knows it)
        \item Other key is public (everyone knows it)
    \end{itemize}
\end{definition}

\subsektion{RSA algorithm}
** We implemented this in hw2 **
\begin{itemize}
    \item Best-known, most used public key algorithm
    \item 1978, Rivest-Shamir-Adleman
\end{itemize}

{\bf How it works:}

To generate an RSA key pair,
\begin{enumerate}
    \item Pick large secret primes $p$, $q$ (randomly chosen, typically 2048
            bits)

        Done by generating odd numbers in range and testing if prime, throwing
        away if not prime and trying again. Primes are dense enough that this
        isn't too bad, and primality testing is also okay in terms of time.
    \item Define $N = pq$

        Useful fact: if $p$, $q$ are prime, for all $0 < x < pq$,
        $$x^{(p-1)(q-1)} \mod{pq} = 1$$
    \item Pick $e$ such that $0 < e < pq$, $e$ relatively prime to $(p-1)(q-1)$
    \item Find $d$ such that $ed \mod{(p-1)(q-1)} = 1$. You can use euclid's algorithm to find $d$.
\end{enumerate}
The public key is $(e, N)$ and the private key is $(d, N)\ [+(p,q)]$.

To encrypt or decrypt with public key:
$$\text{RSA}((e,N), x) = x^e \mod N$$
To encrypt or decrypt with private key:
$$\text{RSA}((d,N), x) = x^d \mod N$$

\begin{theorem*}``It works''
\end{theorem*}
\begin{proof}
    \begin{align*}
    \text{RSA}&((e,N), \text{RSA}((d,N), x))\\
    &= (x^d \mod{pq})^e \mod{pq}\\
    &= x^{de} \mod{pq}\\
    &= x^{a(p-1)(q-1)+1} \mod{pq} \text{, for some $a$}\\
    &= (x^{(p-1)(q-1)})^a x \mod{pq}\\
    &= (x^{(p-1)(q-1)} \mod{pq})^a x \mod{pq}\\
    &= 1^a x \mod{pq}\\
    &= x \mod{pq}\\
    &= x \text{, given $0 < x < pq$}
    \end{align*}
\end{proof}
Best known attack is to try factoring $N$ to get $p$, $q$
\subsektion{Why not use public-key always?}
\begin{itemize}
    \item It's slow ($\sim$1000x slower than symmetric); you're exponentiating huge numbers
    \item Key is big ($\sim$4k bits)
\end{itemize}
\subsektion{How to use public-key crpyto}
For confidentiality: (``your eyes only'')
\begin{itemize}
    \item Encrypt with public key
    \item Decrypt with private key
\end{itemize}
For integrity: (``digital signature'')
\begin{itemize}
    \item ``Sign'' by encrypting with private key
    \item ``Verify'' by decrypting with public key
\end{itemize}
\subsektion{Secure RSA}
!! Warning: Not secure as described above, need to fix !!\\

Problem 1:

Suppose $(e, N) = (3, N)$. Given ciphertext $8$ that was encrypted with $(3, N)$
it's trivial that $x^3 \mod N =8$ has $x = 2$. This shows that you may run into
trouble when encrypting small messages.\\

Problem 2 (Malleability):
\begin{align*}
\text{RSA}((d,N),x) \cdot \text{RSA}((d,N),y) \mod N &= (x^d \mod N)(y^d \mod N) \mod N\\
&= (xy)^d \mod N\\
&= \text{RSA}((d,N), xy)
\end{align*}

$\text{RSA}((d,N), xy)$ is the signature for the message $xy$! Adversary could use this to win the game defining security of the cipher

\begin{definition}
{\bf Malleability}

Adversary can manipulate ciphertext, get predictable result for decrypted
plaintext.

This is usually bad, but sometimes we want a malleable cipher (for some
application)
\end{definition}

Lesser problems:
\begin{itemize}
    \item Same plaintext results in same ciphertext (deterministic)
    \item No built-in integrity check
\end{itemize}

To solve all these problems, add a preprocessing step before encryption. The
standard way is call OAEP (Optimal Asymmetric Encyption Padding):
\begin{enumerate}
    \item Generate 128 bit random value, run through PRG $G$
    \item XOR with message padded with 128 bits of zeros
    \item Run result through PRF $H$, a hash function with announced key
    \item XOR with the random bits
    \item Concatenate result and send to RSA encyption
\end{enumerate}

\makebox[2cm]{}\makebox[1.5cm]{128 bits}\makebox[.5cm]{}\makebox[2cm]{128 bits}\\
\framebox[3.5cm]{\makebox[2cm]{message}\textbar\makebox[1.5cm]{000...}}\makebox[.5cm]{}\framebox[2cm]{random}\\
\makebox[6cm]{\makebox[3.5cm]{$\downarrow$}\makebox[.5cm]{}\makebox[2cm]{$\downarrow$}}\\
\makebox[1.5cm]{}\makebox[.5cm]{$\oplus$}\makebox[2.8cm]{$\longleftarrow\quad G\quad\longleftarrow$}\makebox[.4cm]{$\downarrow$}\\
\makebox[6cm]{\makebox[3.5cm]{$\downarrow$}\makebox[.5cm]{}\makebox[2cm]{$\downarrow$}}\\
\makebox[1.5cm]{}\makebox[.5cm]{$\downarrow$}\makebox[2.8cm]{$\longrightarrow\quad H\quad\longrightarrow$}\makebox[.4cm]{$\oplus$}\\
\makebox[1.5cm]{}\makebox[.5cm]{$\downarrow$}\makebox[2.8cm]{\rule[-0.1cm]{2.8cm}{0.01cm}}\makebox[.4cm]{$\downarrow$}\\
\makebox[1.5cm]{}\makebox[.5cm]{}\makebox[2.8cm]{$\downarrow$}\makebox[.4cm]{}\\
\makebox[1.5cm]{}\makebox[.5cm]{}\makebox[2.8cm]{to RSA encrpytion}\makebox[.4cm]{}\\

Also add the reverse as a postprocessing step after decryption:

\makebox[1.5cm]{}\makebox[.5cm]{}\makebox[2.8cm]{from RSA encrpytion}\makebox[.4cm]{}\\
\makebox[1.5cm]{}\makebox[.5cm]{}\makebox[2.8cm]{$\downarrow$}\makebox[.4cm]{}\\
\makebox[1.5cm]{}\makebox[.5cm]{$\downarrow$}\makebox[2.8cm]{\rule[0.3cm]{2.8cm}{0.01cm}}\makebox[.4cm]{$\downarrow$}\\
\makebox[1.5cm]{}\makebox[.5cm]{$\downarrow$}\makebox[2.8cm]{$\longrightarrow\quad H\quad\longrightarrow$}\makebox[.4cm]{$\oplus$}\\
\makebox[6cm]{\makebox[3.5cm]{$\downarrow$}\makebox[.5cm]{}\makebox[2cm]{$\downarrow$}}\\
\makebox[1.5cm]{}\makebox[.5cm]{$\oplus$}\makebox[2.8cm]{$\longleftarrow\quad G\quad\longleftarrow$}\makebox[.4cm]{$\downarrow$}\\
\makebox[6cm]{\makebox[3.5cm]{$\downarrow$}\makebox[.5cm]{}\makebox[2cm]{$\downarrow$}}\\
\framebox[3.5cm]{\makebox[2cm]{$m'$}\textbar\makebox[1.5cm]{$z'$}}\makebox[.5cm]{}\framebox[2cm]{$r'$}\\

Reject if $z'$ is not all zero, otherwise throw away $r'$ and let $m'$ be the
result of the decyption. $m'$ should at this point be equal to the original
message.

Other things to clean up:
\begin{itemize}
    \item Key size
    \begin{itemize}
        \item To get a big enough key space, need lots of possible primes
        \item Factoring is better than brute force
        \item Factoring algorithms might get better, so build in cushion in key
            size to account for incremental improvements in these algorithms.
        \item Today, 2048-bit primes seem okay
    \end{itemize}
    \item Useful performance trick
    \begin{itemize}
        \item $e = 3$ and make sure $p$ and $q$ are chosen
    such that 3 is relatively prime to $p-1$ and $q-1$
        \item This is extra-big win
    for digital signatures since verify is the common case.
        \item But: what if OAEP disappears from your code?

        Use $e=65537=2^{16} + 1$ instead 
    \end{itemize}
    \item Hybrid crypto: To encrypt a large message,
    \begin{itemize}
        \item Generate random symmetric key $k$
        \item Encrypt $k$ with RSA
        \item Encrypt message with $k$
    \end{itemize}
        Sometimes share the symmetric key using RSA and use that to generate
        further keys to avoid using public-key crypto more than necessary
    \item Hybrid digital signatures: RSA sign(Hash(message))
    \item Claimed identities\\
        Suppose we get a message from ``Alice'' with a digital signature $m^d
        \mod N$. We can verify using $(m^d \mod N)^e \mod N$, but how can we be
        sure of Alice's public key if we don't know Alice?

        Use a digitial certificate (``cert''):
        \begin{itemize}
            \item Bob signs a message saying ``Alice's public key is (...)''
            \item This works if we know Bob and believe him to be trustworthy
                and competent.
            \item If we don't know Bob, then we need to ask Charlie if Bob is
                trustworthy and compentent.
            \item But if we don't know Charlie...
            \item Most common solution: pick universally trustworthy
                ``certificate authority'' who gives out keys\\
        \end{itemize}

        There is also the Web of Trust approach
        \begin{itemize}
            \item Everybody certifies their friends, and if you can find a mutual 
                friend, you're good and people will trust you.
        \end{itemize}
\end{itemize}
}{
%!TEX root = InfoSec.tex
% Lecture 5: 24 September 2014
\sektion{5}{Key Management}
\sidenote{
    US for a long time put restrictions on export of cryptographic software, the
    same restrictions as munitions, requiring a special license.

    Java, for example, would have liked to include crypto along with runtime
    libraries but hard to get license. Possible solutions:
    \begin{itemize}
    \item plugin architecture: could plug-in if they have their own
    \item designed libraries in a way convenient for people who want to
        implement their own crypto (export general purpose math library without
        the export-control issues.
    \end{itemize}
}
\subsektion{How big should keys be?}
A key should be so big an adversary has negligible chance of guessing it.
\begin{itemize}
    \item Watch out for Moore's law: Computers double in speed every 18 months.
        So, you need to add one more bit every 18 months.
    \item For symmetric ciphers, 128 bits is plenty: $2^{128} \approx 10^{39}$,
        so at 1 trillion guesses per second, takes 10 quadrillion times the
        lifetime of the universe.
    \item Need larger for PRF/hash: suppose we're using for digital signature,
        then we're in trouble if adversary finds a ``collision'' ($x_1 \neq x_2$
        s.t. $H(x_1) = H(x_2)$). Finding a collision is more efficient than
        finding key.

        \sidenote{
            {\bf ``Birthday attack'':}

            Generate $2^{b/2}$ items at random, look for collisions in that set
            ($b$ is the bit-length of your hash). Odds are $\sim$50\%.

            Attack requires O($2^{b/2}$) time and O($2^{b/2}$) space, also
            possible in constant space.

            Pepople can generate invalid digital certificates through exploiting
            these collisions.
        }
        Upshot: PRF output size is typically 2x cipher output size to be safe
        (256 bits)
\end{itemize}

\subsektion{Key management principles}
\begin{enumerate}
\setcounter{enumi}{-1}
    \item Key management is the hard part
    \item Keys must be strongly (pseudo)random
    \item Different keys for different purposes (signing/encrypting, encrypting
        vs MACing, Alice to Bob vs Bob to Alice, different protocols)
    \item Vulnerability of a key increases
    \begin{itemize}
        \item the more you use it
        \item the more places you store it
        \item the longer you have it
    \end{itemize}
    So change keys that get ``used up'', and use ``session keys''. If Alice and
    Bob share a long-term key, generate a fresh key just for now and use the
    long-term key to ``handshake'' and agree on which fresh key to use.
    \item The hardest key to compromise is one that's not in accessible storage
        (e.g. a key that's in a drive locked in a safe or stored offline).
    \item Protect yourself against compromise of old keys (forward secrecy);
        destroy keys when you're done with them (and keep track of where the
        keys are)

        For example, it's bad if Alice tells Bob, "Here's our new key, encrypted under the old key." If Mallory records this message and later breaks the old key, she now can also get the new key.
\end{enumerate}

{\bf Diffie-Hellman key exchange (D-H):} 1976\\
Like RSA, relies on a hardness assumption. Here, rely on hardness of ``discrete
log'' problem (given $g^x \mod p$, find $x$). $g,p$ are public, and $p$ is a
large prime.

\begin{table}[h!]
\centering
\begin{tabular}{cccc}
Alice & & Bob & \\
\cline{1-3} & & & \multirow{8}{*}{\begin{sideways}$\xleftarrow{\quad\qquad\text{time}\qquad\quad}$\end{sideways}}\\
& agree on $g,p$ (public), & & \\
& $p = 2q+1$, $q$ prime (``safe prime'') & & \\
random $a$, & & random $b$, & \\
$1 < a < p-1$ & & $1 < b < p-1$ & \\
& $\xrightarrow{g^a \mod p} \xleftarrow{g^b \mod p}$ & & \\
$\left(g^b \mod p\right)^a \mod p$ & & $\left(g^a \mod p\right)^b \mod p$ & \\
$= g^{ba} \mod p$ & & $= g^{ab} \mod p$ &
\end{tabular}
\end{table}

Adversary's best attack is to try to solve the discrete log problem. So Alice
and Bob know something that nobody else knows.

In practice, use $H(g^{ab} \mod p)$ as a shared secret.

BUT: works against an evesdropper (``passive adversary'', ``Eve'') but insecure
if adversary can modify messages (``man in the middle'', ''MITM'' attack). 

\begin{table}[h!]
\centering
\begin{tabular}{ccccc}
Alice & & Mallory & & Bob\\
\hline
$a$ & & $u \qquad v$ & & $b$\\
& $\xrightarrow{g^a \mod p}$ & & $\xleftarrow{g^b \mod p}$ &\\
& $\xleftarrow{g^u \mod p}$ & & $\xrightarrow{g^v \mod p}$ &\\
$g^{au} \mod p$ & $\xleftrightarrow{\qquad\quad}$ & $g^{au} \quad g^{av}$ & $\xleftrightarrow{\qquad\quad}$ & $g^{bv} \mod p$
\end{tabular}
\end{table}

Upshot: D-H gives you a secret shared with \emph{someone}.

Solution:
\begin{enumerate}
    \item Rely on physical proximity or recognition to know who's talking
    \item Consistency check: check that A, B end up with the same value $g^{ab}$
        or that A, B saw the same messages.
\end{enumerate}
How?

Use digital signature (by one party, typically the server)

If Bob can verify Alice's signature, but not the other way around, this still
works (say Alice is a well-known server).

This gives two properties at once:
\begin{itemize}
    \item A authenticates B or vice verse
    \item No MITM, so A and B have a shared secret
\end{itemize}

{\bf D-H and forward secrecy:}\\
Suppose Alice, Bob already have a shared key and want to negotiate a new key.
Then they can do a simple D-H key exchange, protected by old key, then get new
key.

If an adversary doesn't know the old key, can't tamper with the D-H messages.
Even if the adversary gets an old key, not knowing the old key \emph{in real
time} means Mallory can't attack the D-H exchange, and can only be a passive adversary. So Alice and Bob get forward
secrecy with relatively low cost.

Another problem, similar to MITM:

\begin{table}[h!]
\centering
\begin{tabular}{ccccc}
Alice & & Mallory & & Bob\\
\hline
$a$ & & & & $b$\\
& $\xrightarrow{g^a \mod p}$ & & $\xleftarrow{g^b \mod p}$ &\\
& $\xleftarrow{\quad1\quad}$ & & $\xrightarrow{\quad1\quad}$ &\\
$1^a \mod p$ & & & & $1^b \mod p$
\end{tabular}
\end{table}

So abort if receive a 1. Another bad value is $p-1$.

Note: \begin{align*}
\left(p-1\right)^2 \mod p &= \left(p^2 - 2p + 1\right) \mod p\\
    &= (0-0+1) \mod p\\
    &= 1
\end{align*}
Then:
$$\left(p-1\right)^a \mod p = \begin{cases}1 &\mbox{if $a$ is even}\\
    p-1 &\mbox{if $a$ is odd}\end{cases}$$
So also abort if receive $p-1$.

If you chose a safe prime, $1$ and $p-1$ are the only bad values, and there's a
very small chance that one of these would be sent legitimately (plus Alice and
Bob may be checking to make sure they don't send them anyway).\\


\begin{theorem}
If $\frac{p-1}{2}$ is prime, then 1 and $p-1$ are the only bad cases in DH.
So, $p$ is a safe prime if $\frac{p-1}{2}$ is also a prime.
\end{theorem}
}{
\input{Lecture06}}{
\input{Lecture07}}{
% Lecture 8: 10 October 2012
% System and software security - done with Crypto
\sektion{8}{Access control}
How you reason about and enforce rules about who's allowed to do what in the
system. 
\\
\\
Secure system design = secure components + isolation + access control.
\\
\\
This deals with authentication (Who is asking?), not authorization (Does that
person have permission?).
\\
\\
Two authorization approaches:
	\begin{itemize}
	\item access control matrix/list
	\item capabilities
	\end{itemize}

\begin{definition}{Trusted subsystem}\\
A program, with state, that is \underline{isolated} from
the rest of the world, and interacts via \underline{declared interfaces}
\end{definition}

Access control: SUBJECT wants to do VERB on OBECT. Okay?

Policy: a set of (S,V,O) triples that are allowed
\begin{itemize}
    \item How to determine policy? (\underline{should})
    \item How to enforce policy? (\underline{is})
\end{itemize}
One data structure: Access Control Matrix\\
\parbox[c]{5cm}{\makebox[5cm]{$\longleftarrow$ objects $\longrightarrow$}\\
\parbox[c]{1cm}{$\uparrow$\\
\begin{sideways}subjects\end{sideways}\\
$\downarrow$}
\makebox[.5cm]{}
\fbox{$V_1, V_2$}}

\subsektion{Subjects and labels}
\begin{itemize}
    \item subject = some process
    \item Object is some resource (file, open network connection, window)
    \item often, give labels to subjects and set policy based on labels
    e.g. label a process with a user id\\
    (+) reduces matrix size\\
    (+) easier to make policy based on labels\\
    (--) oversimplifies? Suppose: label = userid and means program is running
    ``for'' userid. Alice runs a program written by Bob (example: Alice uses a
        text editor written by Bob to edit Alice's secret file). What label?
    \begin{itemize}
        \item If treat as Alice: Bob's code can send Alice's secret data to Bob
        \item If treat as Bob: Alice can't edit her secret file, can read Bob's files
        \item If treat as Bob but special for this file: none of the labelling
        benefits
        \item If treat as intersection of privileges: get all the drawbacks
    \end{itemize}
    \item Common approach in OS (e.g. Linux): setuid bit
   	 \begin{itemize}
    	\item Bob decides whether program runs as himself or invoker
    	\end{itemize}
\end{itemize}
Store access control info:
\begin{itemize}
    \item as AC matrix - note that this will be very sparse
    \item as ``profiles'' - for each user, list of what subject can do (i.e. row of AC matrix)
    \item as Access Control List (ACL) - for each object, list of (Verb,
            Subject) pairs (who can do what to it). This is typically used
    because small and simple in practice. Often, ACL are stored along with object.
\end{itemize}
Who sets policy?
\begin{itemize}
    \item centralized (``mandatory'') - done by an authority

    (+) done by a well-trained person\\
    (+) might be required (ethical, legal, or contractual obligations)\\
    (--) inflexible, slow
    \item decentralized (``discretionary'') - each object has an
    \underline{owner}, owner set ACLs

    (+) flexible\\
    (--) every user makes security decisions (mistake-prone)
    
    \item mix - owner can choose, within limits set by centralized authority
\end{itemize}
Groups and Roles:\\
Group is a set of people with some logical basis; role is group with one
member\\
Advantages:
\begin{itemize}
    \item makes ACL smaller, easier to understand
    \item change in status naturally causes change in access to resources
    \item ACL encodes reason for access in system (i.e. why you have access)
\end{itemize}
Roles can be hidden temporarily, ``wearing different hats'' (useful for testing)

\subsektion{Traditional Unix File Access}
File belong to one user, one group. 
\\
\\ ACL for each operation contains subset of $\{\text{user}, \text{group}, \text{everyone}\}$. Every VERB requires 3 bits for each operation. 
\\
\\
Every file also has a setuserid bit. 
\begin{itemize}
	\item treat as file owner if setuid = true
	\ item treat as invoker if setuid = false
\end{itemize}

\subsektion{Capabilities}
A different approach to access control: controls access without identification,
like a physical key, ``the bearer has permission to do VERB on OBJECT.''

Sometimes make them revokable, but that's a pain to do in practice

Implementation: crpytographic
\begin{enumerate}
    \item system has a secret key $k$, capability = MAC($k$, verb || object)
    \item public-key: one party grant permission (makes digital signature),
    another party control access (makes sure handed valid capability - verifies
    signature)
\end{enumerate}
Implementation: OS table\\
OS stores a list of your capabilities; Alice makes a system call to give Bob
capabilities for a certain file (file descriptors used to say you've an open
file are an example)

Implementation: in a type-safe programming language (like Java), pointer to an
object is a capability
\\
\\
Tradeoffs:
\begin{itemize}
    \item cryptographic

    (+) totally decentralized \\
    (--) if capability leaks, big trouble\\
    want some kind of revocation, but hard to do
    
    \item OS table
    
    (+) can control flow of passage of capabilities\\
    (+) revocation is much easier\\
    (--) centralized, requires overhead, lack of flexibility, 
\end{itemize}

\subsektion{Logic-based authorization}
Define a formal logic, with primitives for
\begin{itemize}
    \item principals (e.g. users/groups)
    \item objects
    \item delegation
    \item time
\end{itemize}
To get access, submit a proof that you are authorized

Parties make statements by digital signing

System allows for great complexity in policies, but only need simple proof-
checking mechanism to make it work. But also need to work out a way to get
people able to write these statements, and deal with possible large proof size

Caveat: people don't actually use complicated access control mechanisms, and
usually just leave them as the defaults or make it visible to the whole world

Want to come up with a system which infers what the user wants from the way the
user behaves (best if not visible to user)
}{
<<<<<<< HEAD
%!TEX root = InfoSec.tex
% Lecture 9: 8 October 2014
\sektion{9}{Secure Programming}

=======
% Lecture 9: 08 October 2014
\sektion{9}{Information flow and multi-level security}
>>>>>>> 8e525b42b773087a879a1983645b17cc93ee0bb7
Information flow: how to control propagation of information within a program or
between programs on a system where there is some confidentiality requirement.

Consider a program $P(v, s, r)$:
\begin{itemize}
    \item $v$: visible (public) input
    \item $s$: secret input
<<<<<<< HEAD
    \item $r$: randomness (secret)
\end{itemize}
Output: all visible actions of program but doesn't leak secret input

Does the output of $P$ leak information about $s$? Define a game against
adversary where he provides $s_0$ and $s_1$. We announce $P(v, s_b, r)$ where $b \in \{1, 0\}$ and $r$ is secret and random. The adversary guesses what $b$ is. Security is defined by the adversary having no strategy that nets him a a non-negligible advantage in finding $b$.
=======
    \item $r$: random seed
\end{itemize}
Output: all visible actions of program. Output doesn't "leak" s. 

Does the output of $P$ leak information about $s$? Define a game against
adversary guessing between two possible secrets $s$ (similar to semantic security). To avoid leakage, the distribution of outputs must be independent of $s$ for all possible values of $v$.

Game:
\begin{itemize}
	\item adversary chooses v, $s_0$, $s_1$
	\item we announce P(v, $s_b$, r), where $b \in \{0,1\}$, r are secret and random
	\item adversary guesses b
\end{itemize}

We say P doesn't leak s if adversary can't be correct with probability non-negligibly greater than $50\%$ (assuming a computationally-limited adversary).
>>>>>>> 8e525b42b773087a879a1983645b17cc93ee0bb7

How to enforce non-leakiness?

Unlike with previous properties, cannot enforce by watching $P$ run.
(Just because no output came out doesn't mean there wasn't a leak - ``dog that
<<<<<<< HEAD
didn't bark problem''). It's inherently necessary to consider what-ifs that
differ from what you actually saw
=======
didn't bark problem''). We can't prove this property by testing. It's inherently necessary to consider what-ifs that differ from what you actually saw. 

Also, in practice requirement are more complex (more complex labels, and different labels on different data). 

Generalization:
\begin{itemize}
	\item label information (e.g. inputs)
	\item put requirements on outputs
	\item enforce that outputs respect requirements
\end{itemize}
>>>>>>> 8e525b42b773087a879a1983645b17cc93ee0bb7

\subsektion{Lattice model}
General model for information flow policy

<<<<<<< HEAD
\begin{definition}{Lattice} % these are very similiar to binary relations in econ310

    $(S, \sqsubseteq)$, $S$: set of labels, $\sqsubseteq$: partial order such
    that for any $a, b \in S$, there exists a least upper bound $u \in S$ and a
    greatest lower bound $l \in S$.

    least upper bound of $a, b$:
    \begin{itemize}
        \item $a \sqsubseteq u$ and $b \sqsubseteq u$ and for all $v \in S$,
            $a \sqsubseteq v$ and $b \sqsubseteq v$ $\Rightarrow$
            $u \sqsubseteq v$
    \end{itemize}
=======
\begin{definition}{Lattice}

    $(S, \sqsubseteq)$, $S$: set of states, $\sqsubseteq$: partial order such
    that for any $a, b \in S$, there is a least upper bound of $a, b$ and a
    greatest lower bound of $a,b$.
>>>>>>> 8e525b42b773087a879a1983645b17cc93ee0bb7

    partial order:
    \begin{itemize}
        \item reflexive: $a \sqsubseteq a$
        \item transitive: $a \sqsubseteq b$ and $b \sqsubseteq c$, then
            $a \sqsubseteq c$
        \item asymmetric: $a \sqsubseteq b$ and $b \sqsubseteq a$, then $a = b$
    \end{itemize}
<<<<<<< HEAD

=======
    least upper bound of $a, b$:
    \begin{itemize}
        \item $a \sqsubseteq U$ and $b \sqsubseteq U$ and for all $V \in S$,
            $a \sqsubseteq V$ and $b \sqsubseteq V$ $\Rightarrow$
            $U \sqsubseteq V$
    \end{itemize}
    greatest lower bound of $a, b$:
    \begin{itemize}
        \item $L \sqsubseteq a$ and $L \sqsubseteq b$ and for all $V \in S$,
            $V \sqsubseteq a$ and $V \sqsubseteq a$ $\Rightarrow$
            $V \sqsubseteq L$
    \end{itemize}
>>>>>>> 8e525b42b773087a879a1983645b17cc93ee0bb7
\end{definition}
\begin{example}{Lattices}
    \begin{enumerate}
        \item linear chain of labels:

            public $\sqsubseteq$ confidential

            unclassified $\sqsubseteq$ classified $\sqsubseteq$ secret
                $\sqsubseteq$ top secret
<<<<<<< HEAD
        \item compartments (eg. project, client ID, job function)

            label (project) is set of states (project 1, project 2, etc.), $\sqsubseteq$ is subset
        \item org chart

            label is node in chart, $\sqsubseteq$ is ancestor/descendant
        \item combination/cross product of lattices

            label is $(S_1, S_2)$, $(A_1, B_1) \sqsubseteq (A_2, B_2)$ iff
            $A_1 \sqsubseteq A_2$ and $B_1 \sqsubseteq B_2$
    \end{enumerate}
\end{example}
\subsektion{Enforcing flow in a program}
At each point in the program, every variable has a label (that comes from the
lattice we're using). Inputs are tagged with label, outputs are tagged with a requirement. Labels are propagated
when code executes:
$$a = b \Rightarrow \text{label}(a) = \text{label}(b)$$
$$a = b + c \Rightarrow \text{label}(a) = \text{LUB}(\text{label}(b), \text{label}(c))$$
Where LUB = Least Upper Bound. ``Go to a place that's at least as well protected as b and at least as well protected as c.''

Before providing output, check that state of output value is consistent with
policy. Allow the output of a value $v$ where $L$ is required if and only if label($v$) $\sqsubseteq L$

But this isn't enough (only monitoring and rejecting when inconsistent with
policy) -- ``dog that didn't bark''. \textbf{A troublesome example:}
\begin{verbatim}
// label(a) = ``secret''
b = 0;  // 0 isn't secret, so b is set to unclassified
if (a > 5)  b = 1;  // Requires static (compile-time) analysis to get right
=======
        \item compartments (e.g. project, client ID, job function)

            state is set of labels, $\sqsubseteq$ is subset
        \item org chart

            state is node in chart, $\sqsubseteq$ is ancestor/descendant
        \item combination/cross product of lattices

            state is $(S_1, S_2)$, $(A_1, B_1) \sqsubseteq (A_2, B_2)$ iff
            $A_1 \sqsubseteq A_2$ and $B_1 \sqsubseteq B_2$
    \end{enumerate}
\end{example}
\subsektion{Information flow in a program}
At each point in the program, every variable has a state/label (that comes from the
lattice we're using). Inputs are tagged with state. Outputs are tagged with a requirement. States are propagated
when code executes.

Example: $a = b + c$; State($a$) = LUB(State($b$), State($c$))
[LUB = Least Upper Bound]

Before providing output, check that state of output value is consistent with
policy. (For example, only allowed to emit unclassified output.) Formally, ensure that $label(v) \sqsubseteq L$, where L is the required policy.

But this isn't enough (only monitoring and rejecting when inconsistent with
policy) -- ``dog that didn't bark'':
\begin{verbatim}
// State(a) = ``secret''
// State(c) = State(d) = public
b = c;
if (a > 5)  b = d;  // Requires static (compile-time) analysis to get right
>>>>>>> 8e525b42b773087a879a1983645b17cc93ee0bb7
output b;   // Says something about a, so should be labelled as secret
\end{verbatim}
Static analysis won't catch all, but will catch some of the leaks.
\begin{description}
    \item[Problem 1:] conservative analysis leads to being overly cautious
<<<<<<< HEAD
    \item[Problem 2:] timing might depend on values
    \item[Problem 3:] ``covert channels''; shared resource channels might reveal clues about, for example, what files are being opened and what secret(s) that might affect.
=======
    \item[Problem 2:] timing might depend on values (can lead to covert channel attacks)
>>>>>>> 8e525b42b773087a879a1983645b17cc93ee0bb7
\end{description}

{\bf What if you can't prevent a program from leaking the information it has?}

<<<<<<< HEAD
{\bf Bell-LaPadula model}: lattice-based information flow for programs and files
\begin{itemize}
\item every program has a label (from lattice): what it's allowed to access
\item every file has a label: what it contains
\item Rule 1: ``No Read Up'' - Program $P$ can read File $F$ only if label($F$)
    $\sqsubseteq$ label($P$)
\item Rule 2: ``No Write Down'' - Program $P$ can write File $F$ only if
    label($P$) $\sqsubseteq$ label($F$); Programs can't talk to ``lower'' files so as
    not to leak information
\end{itemize}
\begin{theorem*} If label($F_1$) $\sqsubseteq$ label($F_2$) and the two rules are
=======
Conservative assumption (contagion model): every programs leaks all its inputs to all its outputs. 

{\bf Bell-LaPadula model}: lattice-based information flow for programs and files
\begin{itemize}
\item every program has a state (from lattice): what it's allowed to access
\item every file has a state: what it contains
\item Rule 1: ``No Read Up'' - Program $P$ can read File $F$ only if State($F$)
    $\sqsubseteq$ State($P$)
\item Rule 2: ``No Write Down'' - Program $P$ can write File $F$ only if
    State($P$) $\sqsubseteq$ State($F$)
\end{itemize}
\begin{theorem*} If State($F_1$) $\sqsubseteq$ State($F_2$) and the two rules are
>>>>>>> 8e525b42b773087a879a1983645b17cc93ee0bb7
enforced, then information from $F_2$ cannot leak into $F_1$.
\end{theorem*}

Problems:
\begin{enumerate}
<<<<<<< HEAD
    \item expections (need to make explicit loopholes in system to allow)
    \begin{itemize}
        \item declassification of old data
=======
    \item exceptions (need to make explicit loopholes in system to allow)
    \begin{itemize}
        \item declassify/unprotect old data
        \item what about encryption (hope "secret" ciphertext doesn't leak plaintext)
>>>>>>> 8e525b42b773087a879a1983645b17cc93ee0bb7
        \item aggregate/``anonymized'' data
        \item policy decision to make exception
    \end{itemize}
    \item usability - system can't tell you if there are classified files in a
        directory you're trying to delete or no space on disk for you to add a
        file
    \item outside channels - people talk to each other outside the system
\end{enumerate}

This, so far, has been about confidentiality. Can we do the same thing for
integrity?
<<<<<<< HEAD

\begin{example}
Any program is able to delete a file and overwrite ``secret plans''. Our program/file
is public, so it can write UP and replace contents of ``secret plans''. This is bad.
\end{example}

{\bf Biba model}: (B-LP for integrity)
\begin{itemize}
=======
\begin{itemize}
    \item State: level of trust in integrity of information
    \item ensure high-integrity data doesn't depend on low-integrity inputs
        (try to avoid GIGO problem)
\end{itemize}

{\bf Biba model}: (B-LP for integrity)
\begin{itemize}
    \item Label/state: how much we trust program with respect to integrity/how important file is
>>>>>>> 8e525b42b773087a879a1983645b17cc93ee0bb7
    \item Rule 1: ``No Read Down''
    \item Rule 2: ``No Write Up''
\end{itemize}

B-LP model and Biba model at the same time?
\begin{itemize}
    \item if use same labels for both (high confidentiality = high integrity),
        then no communication between levels
    \item if different labels, then some information flows become possible, but
        could result in being much more difficult for users
<<<<<<< HEAD
    \item especially with static program analysis, things become conservative
        and flexibility is lost
    \item result: usually focus on confidentiality or integrity and let humans
        worry about this outside of the system
\end{itemize}

The takeaway is that information flow tools are useful for preventing yourself from
making mistakes, but not so useful to protect against an adversary.

\sidenote{{\bf Back to crypto... [a note from 2012]}
=======
    \item result: usually focus on confidentiality or integrity and let humans
        worry about this outside of the system
\end{itemize}
\sidenote{{\bf Back to crypto...}
>>>>>>> 8e525b42b773087a879a1983645b17cc93ee0bb7

Secret sharing:
\begin{itemize}
    \item divide a secret into ``shares'' so that all share are required to
        reconstruct secret
        \begin{itemize}
            \item 2-way: pick a large value $M$, secret is some $s$,
                $0 \le s < M$\\
                pick $r$ randomly, $0 \le r < M$\\
                shares are $r$, $(s-r) \mod M$\\
                to reconstruct, add shares $\mod M$
            \item $k$-way: shares $r_0, r_1, \dots, r_{k-2}$ random,
                $(s - (r_0 + \cdots + r_{l-2})) \mod M$
            \item can also construct degree $k$ polynomials such that $k$ values
                are needed to reconstruct
        \end{itemize}
    \item suppose RSA private key is $(d, N)$, shares
        $(d_1, N), (d_2, N), (d_3, N)$ such that
        $d_1 + d_2 + d_3 = d \mod(p-1)(q-1)$

        $\left(X^{d_1} \mod N\right)\left(X^{d_2}\right)\left(X^{d_3}\right) =
        X^{(d_1 + d_2 + d_3)\mod(p-1)(q-1)} \mod N = X^d \mod N$

        (splits up an RSA operation)
\end{itemize}
}
}{
\input{Lecture10}}{
% Lecture 10: 10 November 2014
\sektion{11}{Spam}
Focus on email.

Scope of problem:
\begin{itemize}
    \item Vast majority of email is spam (99+\%)
    \item Lots is fraudulent (or inappropriate)
    \item 5\% of US users have bought something from a spammer

        The anonymity makes this attractive for certain kinds of products
    \item Spamming often pays (low cost to send, so need little success to
            profit)
\end{itemize}

\sidenote{{\bf Review: how email works}
\begin{itemize}
    \item Messages written in standard format
        \begin{itemize}
            \item Headers: To, From, Date, ...
            \item Body: can encode different media types in body
        \end{itemize}
    \item Traditionally:

        \framebox{\parbox[c]{2cm}{sender's computer}} $\xrightarrow{\text{SMTP}}$
        \framebox{\parbox[c]{2cm}{sender's MTA}} $\xrightarrow{\text{SMTP}}$
        \framebox{\parbox[c]{2cm}{recipient's MTA}} $\xrightarrow{\text{IMAP}}$
        \framebox{\parbox[c]{2cm}{recipient's computer}}

        (MTA: Mail Transfer Agent)
    \item Webmail model:

        \framebox{\parbox[c]{2cm}{sender's computer}} $\xrightarrow{\text{HTTP(S)}}$
        \framebox{\parbox[c]{2cm}{sender's mail\\service}} $\xrightarrow{\text{SMTP}}$
        \framebox{\parbox[c]{2cm}{recipient's mail\\service}} $\xrightarrow{\text{HTTP(S)}}$
        \framebox{\parbox[c]{2cm}{recipient's computer}}
    \item More complexities:
        \begin{itemize}
            \item Forwarding
            \item Mailing lists
            \item Autoresponders
        \end{itemize}
\end{itemize}
}

\subsektion{Economics of spam}
It is very cheap to send email.

Most of the cost falls on recipient
\begin{itemize}
    \item Needs to store message
    \item Human takes the time to actually read the message
\end{itemize}
Sender will send when sender's cost $<$ sender's benefit. 
What we would like: send when total cost $<$ total benefit.

\subsektion{Anti-spam strategies}
Laws, technology, or combination

Note that most spam is already illegal (either fraudulent offer, false
    adverising, or offering an illegal product).

\begin{definition}{Spam}
\begin{enumerate}
    \item Email the recipient doesn't want to receive

        Problems:
        \begin{itemize}
            \item Defined after the fact
            \item Legally problematic (anyone can say they didn't want some
                    message)
            \item Not what you want (just not wanting it doesn't make it spam)
        \end{itemize}
    \item Unsolicited email

        Problems:
        \begin{itemize}
            \item What does this mean? (May not explicitly have asked for a
                    given email)
            \item Lots of unsolicited email is wanted
        \end{itemize}
    \item Unsolicited commercial email

        Problems: less than in definition 2, but still the same issues
\end{enumerate}
\end{definition}

{\bf Free speech:} (legal constraint, principle we'd like to honor)
\begin{itemize}
    \item Minimum: don't stop a communication if both parties want it
    \item Legally, there's no absolute right not to hear undesired speech.
    \item Commercial speech gets less protection than political speech.
\end{itemize}

\begin{definition}{Spam (CAN SPAM Act)}

Any commercial, non-political email is spam unless:
\begin{enumerate}[(a)]
   \item it's non-commercial, or
   \item it's political speech, or
    \item recipient has explicitly consented to receive it, or
    \item sender has a continuing business relationship with recipient, or
    \item email relates to an ongoing commercial transaction between the sender
        and receiver
\end{enumerate}
\end{definition}
Note: this definition is not testable in software.

Vigorous enforcement against wire fraud, false medical claims, etc. (can follow
the money)

Law against forging the From address is surprisingly effective\\
Disadvantages for spammer:
\begin{itemize}
    \item Forced to identify themself
    \item Easy to filter
\end{itemize}

Private lawsuits
\begin{itemize}
    \item ISP sue spammer to cover their server resources, etc.

    (AOL: sue spammers and give a random user their stuff!)
    \item Serves as a deterrent (expensive to bring lawsuit)
\end{itemize}

Laws have succeeded in drawing a line between spam and legitimate businesses. 

Anti-spam technologies:
\begin{itemize}
    \item Blacklist:

        List of ``known spammers'', refuse to accept mail from them
        \begin{itemize}
            \item If list holds From addresses: Spammer will spoof From address
                (then spammer is also breaking the law)
            \item If list holds IP addresses: Spammers move around, compromise
                innocent users' machines and send spam from there (very common);
                also note that outgoing email IP address is often shared
        \end{itemize}
        How aggressive should you be about putting people on the blacklist?
        \begin{itemize}
            \item Too slow: spammers can get away with spamming
            \item Too quick: false blocking
                %(Felten's domain blocked on SpamCop's list of banned IP
                 %addresses, got cancelled by webhosting company for policy
                 %violation -- had never sent an email from that domain, so had a
                 %good argument against being a spammer; had written a blog post,
                 %and someone linked to that post then sent it to a friend who
                 %marked it as spam and couldn't unreport; blacklisted 2 weeks)
        \end{itemize}
        Denial of service: forge spam "from" victim
    \item Whitelist:

        List of people you know, reject email from everybody else
        \begin{itemize}
            \item Blocks too much
            \item But: can combine with other countermeasures (exempt certain
                    people from another countermeasure)
        \end{itemize}
    \item Require payment (Pay-to-send):
        \begin{itemize}
            \item Pay in money:

                Sender pays receiver\\
                OR sender pays receiver IF receiver reports messsage as spam
                (generates incentive problem for reciever)\\
                OR sender pays charity if receiver reports as spam

                Problem: really expensive for large mailing lists
            \item Pay in wasted computing time:

                Sender must solve some difficult computational puzzle

                Works internationally, but big problem for large mailing lists,
                destroys computing time
            \item Pay in human attention:

                CAPTCHA %(Completely Automated Public Turing test to tell
                        %Computers and Humans Apart)

                Can hire solving of CAPTCHAs in various ways (sweatshops, make
                    people solve to see porn, ...)
                 
             \item Pros and Cons
                		 \begin{enumerate}[(a)]
  			 \item raises cost of spamming (+)
  			 \item often raises cost of legit mail (-)
  			 \item often wastes resources rather than transferring them (-)
			\end{enumerate}
        \end{itemize}
    \item Sender authentication
    	\begin{itemize}
			\item address-based (e.g. Princeton says which IPs can send @princeton.edu From address)
			\item signature by domain owner
	\end{itemize}
    \item Content-based filtering:

        Recipient applies filter algorithm ot content of incoming email
        \begin{itemize}
            \item Early days: keyword-based

                Lots of false positives, ways to work around these
            \item Now: word-based machine learning, often also personalized
                (relying on user to label as spam or not)

                Spammer can test the non-personalized part by making an account
                and seeing what gets marked as spam -- until filters started
                looking for the word-salad test messages
            \item Collaborative filtering: use other users' reports as indicator
                of spam
        \end{itemize}
\end{itemize}

Robocalls now a huge problem: FTC gave public challenge to solving this
}{
%\input{Lecture12}}{ % See lecture slides
%\input{Lecture13}}{ % See lecture slides
\input{Lecture14}}{
\input{Lecture15}}{
<<<<<<< HEAD
%!TEX root = InfoSec.tex
% Lecture 21: 1 December 2014
\sektion{21}{Economics of security}

\textbf{Does the market produce optimal security?}\\
\textbf{What is optimal?}

\begin{enumerate}
	\item Definition 1: Strong Pareto efficiency

	\begin{itemize}
		\item Condition A is strong Pareto superior to condition B if everyone likes A better than B
		\item Condition is strong Pareto efficient if no strong Pareto superior alternative is available
		\item Criticisms: requires that everyone agrees that condition A is better
	\end{itemize}
	
	\item Kaldor-Hicks efficiency

	\begin{itemize}
		\item Less stringent than Pareto efficiency (which requires that no one is worse off)
		\item  Condition A is KH superior to condition B if there exists zero-sum payments P among people such that A + payments is strong Pareto superior to B
	(payments need not happen, theoretical construct)
		\item For example, if the beneficiaries of pollution could theoretically pay the victims enough that neither party is worse off, that's KH efficient
		\item Criticisms: payments are theoretical. So taking \$1 from every poor person and giving \$1.02 each to Bill Gates is KH efficient
	\end{itemize}
\end{enumerate}

\textbf{A world with perfect information and perfect bargaining} would be SP efficient and KH efficient. 

\textit{Proof:}\\
\textbf{SP efficiency:} By contradiction: \\
	Suppose outcome is not SP efficient. Then an alternative exists that is SP superior to outcome. Then bargaining would lead to that alternative.

\textbf{KH efficiency:} also by contradiction:
	Suppose outcome is not KH efficient. Then there is an alternative A, payments P that A + P is SP superior to outcome. Therefore, outcome is not KH-efficient.

So, there must be some market failure happening because the world is certainly not SP efficient (and thus certainly not KH efficient). \textit{Note that the goal is not maximum security, but efficent security. Invest in a solution only if TOTAL BENEFIT $>$ TOTAL COST}

\subsektion{Market Failure 1}
\textit{Negative extenalities} (think spam, DDoS). The user will invest in reducing harm to self, but not in reducing harm to strangers. So here, there is an underinvestment in security because there is an external harm (beyond producer to buyer), and bargaining to fix externalities is not possible in the real world.

\subsektion{Market Failure 2}
\textit{Asymmetric information:} It's hard for buyers to evaluation the security of products. The producer knows more about the security of the product than the customers do.

Recall the ``lemons market'' from a few lectures ago. If a consumer can't tell high quality goods from low quality goods, the consumer won't pay extra for high quality and the producer then won't invest in quality. 

\textbf{Antidotes} 
\begin{itemize}
	\item warranties
	\item seller reputation
\end{itemize}
(as a sidenote, both work poorly for startup companies)

\subsektion{Network effects}
A product that becomes more valuable as more people use it (think email, phone, search engine) tends to push markets towards monopoly. Standards can lead to positive network effects without monopoly. 

\sidenote{
	\textbf{Network effect cons}\\
	It leads to a monoculture which can help the bad guys\\

	\textbf{Network effect pros}
	\begin{itemize}
		\item scale efficiencies in terms of security
		\item internalize some of the benefits
		\item antidotes to the lemons market problem with be more effective
	\end{itemize}	
}


\textbf{Race to market}: Network effect markets will often tip toward the early leader. There might be lots of contenders for a market, but really only one winner. So, time to market is critical. 

This leads to companies getting an MVP into the market as soon as possible and without waiting for better security. They can make a small upfront payment now in hopes of a large payoff later. This leads to a ``bolt on security'' kind of approach. 

\textbf{Can this be fixed?}
\begin{itemize}
	\item Large customers can protect themselves. There might also be market structures to improve information flow? For example, insurance companies or certification programs.
	\item Liability rules can changed so that a producer must pay user if their product caused a breach. \textit{Optimal liability rule: cost borne by whoever can best prevent harm}. This approach argues for liability for producers, BUT it's (1) hard to attribute blame, (2) hard to measure harm, and (3) there's a substantial cost to adjudication.
	\item Public inspections; a large buyer demands ability to publicize their security evaluations of products 
\end{itemize}

=======
% Lecture 21: 1 December 2014
\sektion{21}{Economics of Security}
Does the market produce optimal security? To understand this question, we'll first want to to define what "optimal" means. 

\subsektion{Definitions of Efficiency}

Definition 1: Strong Pareto Efficiency
\begin{itemize}
    \item Condition A is Strong-Pareto-Superior to Condition B if everyone likes A better than B
    \item "Available alternative" $\implies$ one that is reasonably feasible
    \item A condition is SP-efficient if: no SP-superior alternative is available
\end{itemize}
Definition 2: Kaldor-Hicks efficiency
\begin{itemize}
    \item Condition A is KH-superior to Condition B if there exist zero-sum payments among people such that A + P is SP-superior to B
    \item payments need not happen, just a theoretical construct
    \item A condition is KH-efficient if: no KH-superior alternative is available
\end{itemize}
Using these definitions, consider a world with \emph{perfect information} and \emph{perfect bargaining}. Theorem: Outcomes in this world will be both SP-efficient and KH-efficient. 
\\
Proof:
\begin{itemize}
	\item SP-Efficient: By contradiction. Assume the outcome is not SP-Efficient. Then an alternative exists that is SP-superior to the existing outcome. However, bargaining would lead to the adoption of this superior outcome.
	\item KH-Efficient: By contradiction. Assume the outcome is not KH-Efficient. Then there is an alternative A and a set of payments P such that A + P is SP-superior to the existing outcome. Then the outcome isn't SP-Efficient (as proved above).
\end{itemize}
Therefore, if we observed market failures, they must result from a breakdown in either perfect information or perfect bargaining.

\subsektion{Market Failures}
Market failure \#1: Negative Externalities
	\begin{itemize}
		\item If my machine is compromised, some harm falls on me, and some harm falls on strangers (e.g. from denial of service attacks or spam launched from my computer)
		\item In general, users will invest in reducing harm to themselves, but not to strangers
		\item Outcome is underinvestment in security, since the external harm doesn't enter into user's cost/benefit calculation
		\item Breakdown in perfect bargaining, since the "strangers" are unidentified and unable to invest to prevent the harm that falls on them
	\end{itemize}
	
Market failure \#2: Asymmetric Information
	\begin{itemize}
		\item Arises when vendors know more than users about the security of their products
		\item If it is hard for buyers to evaluate the security of products, then they won't be able to differentiate between high-quality and low-quality products
		\item As a result, users won't pay more for supposedly high-quality products, so producers won't invest to develop more secure products, leading to underinvestment in security
		\item Antidotes:
			\begin{itemize}
				\item Warranties: act as a signal of quality to buyers if companies willing to bear the downside of security breaches
				\item Seller reputation: companies may be harmed in the long-run by selling poor quality products due to damaged reputation
				\item Note: both of these solutions don't work well for start-ups, since the lifetime of these companies are short and thus warranties aren't very valuable and seller reputation isn't a large concern
			\end{itemize}
	\end{itemize}
	
Network Effect:
	\begin{itemize}
		\item Some products tend to become more valuable the more people use it (e.g. search engines)
		\item Markets for these products tend to be pushed towards monopoly
		\item Standardization can lead to positive network effect without monopoly
		\item Argument: network effect $\to$ monoculture
		\item Example: if all products use the same security protocol, might be easier for bad guys to break lots of system by exploiting a vulnerability in that standard
		\item However, there are benefits to having a dominant producer of security:
			\begin{itemize}
				\item There are scale efficiencies in security, since large companies can amortize investments over a large number of users
				\item Companies can also internalize some of the security benefits, if users harmed (as in the Negative Externality scenario) fall within the same user base
				\item Antidotes to asymmetric information are more effective (reputation is more important to large companies than start-ups)
			\end{itemize}
	\end{itemize}
	
Race to market:
	\begin{itemize}
		\item Because of the network effect, companies have a strong incentive to gobble up market share as fast as possible
		\item Often, minimum viable products tend not to require large investment in security
		\item Start-ups face decision to invest \$1 in security today and receive a pay-off of \$N in the future
		\item Lead to a "bolt on security" approach, where security features are added once product is already being used
	\end{itemize}
	
\subsektion{Solutions to Market Failures}
Large customers tend to be able to protect themselves; for example, they can demand that certain security features be implemented in a product. But what about individual users?
\\
\\
Can market structures improve information flow? \emph{Insurance companies} (i.e. that offer insurance against security breaches) can aggregate the bargaining power of many different customers. \emph{Certification programs}, which would give products/companies certificates of quality, could lead to the same effect. Presumably, certified companies would see more demand and be able to charge higher prices for their products. However, companies are unlikely to pay certification bodies to criticize their software. 
\\
\\
Can we change liability rules? An optimal liability rule: costs should be borne by whoever can best prevent harm. 
\\
\\ Case study: ATM Fraud
	\begin{itemize}
		\item In the early days of ATMs, many people would withdraw money and claim they didn't to force banks to re-credit their account
		\item In the US, if there wasn't conclusive proof, banks bore the cost
		\item In the UK, if there wasn't conclusive proof, customers bore the cost
		\item level of fraud significantly lower in US, since banks had made investments to gather evidence of withdrawal in order to avoid losses
		\item Generalizing, this seems like an argument in favor of shifting liability for security flaws to producers, since they are generally in a better position to identify and fix errors in software and hardware
	\end{itemize}

Some problems with shifting liability:
	\begin{itemize}
		\item It's hard to attribute blame: e.g. identifying the true source of a denial of service attack launched from a network of computers in many geographic locations
		\item It's hard to measure harm: difficult to isolate the harm caused by a single security breach
		\item There's a substantial cost to adjudication
	\end{itemize}
>>>>>>> 8e525b42b773087a879a1983645b17cc93ee0bb7

}{
% Lecture 22: 3 December 2014
\sektion{22}{Human Factors in Security}
We often attribute security failures to "user error." But why do users make so many errors? There are several sources:
\begin{itemize}
    \item Bad UI / Poor User Experience (UX)
    \item Rational ignorance: sometimes, investing the effort to learn how to use security software or a protocol outweighs the benefit
    \item Heuristic decision making
    \item Cognitive biases
\end{itemize}

Relying on a "smart user" has hidden costs. An example is relying on users to detect "spear-fishing emails" and distinguishing malicious attachments from genuine ones. The user might then have to contact colleagues to verify legitimate emails, costing both parties time. Some emails falsely thought to be malicious might never be opened. 
\\
\\
A common mistake when designing security software is to "design for yourself." There are many different kinds of users, and their needs will probably vary over time, so designing software for the use of a knowledgable developer is probably a misguided approach. 

\subsektion{Example: Wifi Encryption}
\begin{itemize}
    \item Everybody recommends encrypting WiFi networks, but a relatively small number of networks are actually encrypted (e.g. powerless)
    \item Problem: key distribution
    	\begin{itemize}
		\item Key must be known to all devices on the network
		\item Difficult to make the network open to outsiders
		\item Even if we allow everyone onto the network, it seems silly not to encrypt traffic once people are actually connected to the network
	\end{itemize}
    \item Why don't people encrypt?
       	\begin{itemize}
		\item Bad out-of-box experience (people can't access internet immediately)
		\item Some internet-enabled devices don't have I/O (e.g. thermostat, coffee-maker)
		\item The devices need to remember the key over time
		\item A secret key known to 15,000 people isn't really a secret
	\end{itemize}
   \item How can we fix this?
         \begin{itemize}
		\item Exploit physical proximity (e.g. "tap to pair device")
		\item Physical transfer of key through dongle
		\item Adopt a "Trust On First Use" (TOFU) policy, which assumes no impersonation the first time a machine connects
		\item Warning-based approach: warn administrator when a new party connects to the network
	\end{itemize}
\end{itemize}

\subsektion{Case study: Email encryption ("Why Johnny Can't Encrypt")}
Some researchers presented "average users" with a PGP mail client, and asked them to perform tasks that required encryption (e.g. send a secure email to Alice, set up a new secure communication with Charlie). The goals were to observe a) how they use it and b) what mistakes they make. 
\\
\\
The study revealed that average users experienced LOTS of usability problems and made LOTS of security errors. 
Why?
\begin{itemize}
	\item UI design mistakes (e.g. hard to find something in a dropdown menu)
	\item Metaphor mismatch
		\begin{itemize}
			\item e.g. RSA key was visualized with a physical key icon
			\item but a cryptographic key isn't much like a physical key
			\item why would a user want to publish their key? (as they do with the public key) why are there two keys (public and private)?
			\item one suggestion: "ciphertext is a locked chest" (not a perfect analogy)
		\end{itemize}
	\item User has to do lots of work up front, before communicating at all
		\begin{itemize}
			\item have to first generate a key pair
			\item this is the point in the process where users understand the least, and are most eager to send a message
		\end{itemize}
\end{itemize}

This case study raises the question about what role the user should play in a secure procedure. Should they
	\begin{itemize}
		\item control a mechanism? (e.g. "block cookies" on a browser)
		\item use a tool? (e.g. "clear history" on a browser, which performs many tasks like clearing cache)
		\item state a goal?
	\end{itemize}
	
The goal for many systems is a "naturally secure interface." One example here is the camera light on a laptop that is intended to light up whenever camera is in use.
	\begin{itemize}
		\item user obtains protection against being secretly recorded
		\item however, on Macs, this light can be bypassed by re-programming firmware that links camera and light
		\item a better solution would be to build in a \emph{hardware} interlock between camera and light
	\end{itemize}

\subsektion{Social Barriers to Adoption}
Case study: an organization that recruits volunteers to break the law in order to put political pressure on issues. Organizations like this have a strong incentive to encrypt, since their threat model includes an adversary (government) with a large amount of resources. 
\\
\\ Why don't people encrypt more often?
	\begin{itemize}
		\item Stigma attached to use of encryption
			\begin{itemize}
				\item only "paranoid" people use encryption
			\end{itemize}
		\item Etiquette of encryption
			\begin{itemize}
				\item reply should be encrypted if and only if original message was encrypted
				\item seen as impolite to respond to an unencrypted message with an encrypted message, particularly if replying to a superior
			\end{itemize}
		\item Encryption is seen as a barrier to recruitment
			\begin{itemize}
				\item the up-front cost of setting up a dedicated email client might discourage volunteers from participating
			\end{itemize}
	\end{itemize}

\subsektion{Warning messages}
Warning messages are often used to preempt security breaches. However, users can suffer from "dialogue fatigue" and either click through important warnings or find workarounds. 
\\
\\
Countermeasures:
	\begin{itemize}
		\item Vary design of dialogue
		\item Make "No" the default (so user can't click enter to move through0
		\item Delay activation of "OK" button
		\item "Hey, you really need to read this" approach
	\end{itemize}
Lately, designers often choose secure defaults and don't ask the user for an up-front choice. 

\subsektion{NEAT/SPRUCE Framework}
A set of questions created by Microsoft to guide developers:
\\
\\
NEAT: Is your security/privacy UX:
	\begin{itemize}
		\item Necessary? Can you eliminate it or defer user decision?
		\item Explained? Do you present all info user needs to make decision? Is it SPRUCE?
		\item Actionable? Is there a set of steps user can follow to make correct decision?
		\item Tested? Is it NEAT for all scenarios, both benign and malicious?
	\end{itemize}
SPRUCE: Why presenting a choice to user, consider
	\begin{itemize}
		\item Source: say who is asking for decision (which application / component / machine)
		\item Process: give user actionable steps to a good decision
		\item Risk: explain what bad thing could happen if user makes a wrong decision
		\item Unique knowledge: tell user what info they bring to the decision
		\item Choices: list available options, clearly recommend one
		\item Evidence: highlight info user should include/exclude in making the decision
	\end{itemize}
	}{
% Lecture 23: 8 December 2014
\sektion{23}{Quantum Computing}
\subsektion{Classical Bits}
\begin{itemize}
    \item Two states, written as $|0>$ and $|1>$ (classical 0 and classical 1)
    \item Can read bit to recover state
    \item Two single bit logic gates: \emph{identity} and \emph{invert}
\end{itemize}
\subsektion{Quantum Bits (qubits)}
By analogy with classical bits:
\begin{itemize}
    \item State is a linear combination of classical states: $x|0> + y|1>$, such that $|x|^2 + |y|^2 = 1$
    \item x and y are complex numbers (giving the linear combination 4 degrees of freedom)
    \item If x and y are restricted to real values, then possible states represented by the unit circle
    \item Special measurement operation: 
    	\begin{itemize}
		\item Only way to read a qubit's state
		\item Sets the bit to $|0>$ with probability $|x|^2$ and $|1>$ with probability $|y|^2$, and returns the result
		\item "destroys" information by reducing qubit to a classical state
	\end{itemize}
    \item operations on a qubit = any operation that preserves the condition that $|x|^2 + |y|^2 = 1$
    \item writing a qubit as $\begin{pmatrix} x \\ y \end{pmatrix}$, a valid operation on a qubit is any unitary matrix M, since $M\begin{pmatrix} x \\ y \end{pmatrix} = \begin{pmatrix} x' \\ y' \end{pmatrix}$ where $|x'|^2 + |y'|^2 = 1$
    \item example: $R =  \begin{pmatrix} \frac{1}{\sqrt{2}} \ \ \ 0 \\ 0 \ \ \ \frac{1}{\sqrt{2}} \end{pmatrix}$ "rotates" qubit counterclockwise on the unit circle
\end{itemize}

\sidenote{{\bf Tempting (but wrong) view:}

It's tempting to think of a qubit as a classical bit whose state is "unknown" (i.e. is $|0>$ with probability $|x|^2$ and $|1>$ with probability $|y|^2.$ The R operator above is then thought of as a "randomizing" operator. However, applying $R(R(|0>))$ gives $|1>$ with certainty. This doesn't seem to consist with the view that $R(|0>)$ is a "random" bit,  since applying a randomizing operator to randomness shouldn't lead to a certain outcome. 
}

\subsektion{Multi-qubit systems:}
\begin{itemize}
	\item In a classical system with k bits, the possible states of the system are $|\text{(every k-bit string)}>$, of which there are $2^k$
	\item In a quantum system, the possible states are $\sum \limits_{i = 0}^{2^k-1} \alpha_i \cdot |s_i>$, where $s_i$ is the ith k-bit string, such that $\sum \limits_{i = 0}^{2^k-1} |\alpha_i|^2 = 1$
	\item Measuring the system forces it into classical state $|s_i>$ with probability $|\alpha_i|^2$
	\item In principle, operations on the multi-qubit system include any unitary matrix
	\item In practice, we use a few simple gates
	\item Computation:
		\begin{itemize}
			\item initial state $= |(input)0000�0>$ (input is padded with zeros to some fixed length)
			\item perform some preprogrammed set of gates
			\item measure system to produce output
			\item goal is for the system to collapse to the "correct" answer with high probability upon measurement
		\end{itemize}
\end{itemize}

Can we built a quantum computer? Right now, we can only build small ones, but there is reason to believe we will be able to construct larger ones in the future. 

\subsektion{Advantages of QC}
There is a common but wrong view that quantum computers will be able to solve any problem in NP efficiently (i.e. in polynomial time). The idea here is that you can put a quantum computer into a state which is a superposition of all the possible solutions, and then measure to determine the "correct" answer. In reality, decreasing the coefficients on incorrect answers and increasing the coefficient on correct answer(s) isn't always easy. 
\\
\\
What we know:
\begin{itemize}
	\item for general NP hard problems, classical computers require brute force search (takes $O(2^n)$ time)
	\item Grover's algorithm $\implies$ quantum computers can solve these problems in $O(2^{n/2})$, a dramatic improvement but still super-polynomial
	\item there exist some such problems which quantum computers can solve efficiently
		\begin{itemize}
			\item factoring (via Shor's algorithm)
			\item discrete log problem (via a variant of Shor's algorithm)
		\end{itemize}
\end{itemize}

\subsektion{Implications for crypto}
\begin{itemize}
	\item in a world with large quantum computers, protocols that rely on the hardness of factoring and the discrete log problem are useless (including RSA, Diffie-Hellman, etc)
	\item most symmetric crypto (e.g. AES) is not breakable (in the domain of Grover�s algorithm)
	\item Grover's algorithm implies that we can break AES that uses a 128-bit key in $2^{128/2} = 2^64$ steps
	\item Solution: double the key size
	\item are there public key algorithms that are not breakable fast by quantum computers? probably
\end{itemize}

\subsektion{Quantum Key Exchange}
\begin{itemize}
	\item Threat model:
		\begin{itemize}
			\item quantum channel between Alice and Bob
			\item assume there exists an eavesdropper who a) can measure but b) cannot modify qubits in the channel without measuring
		\end{itemize}
	\item 1) before sending, Alice flips coin; if heads, apply R to the bit before sending
	\item 2) when Bob gets a bit, flip coin: if heads, apply $R^{-1}$ to the bit before measuring
	\item 3) Alice and Bob publish their coin flips, discard any bits where flips don�t match
	\item 4) Alice picks half of the remaining bits, at random, and Alice and Bob publish their values for these bits
		\begin{itemize}
			\item if any fail to match, abort the protocol (eavesdropper was measuring)
			\item otherwise, use the remaining bits as a shared secret
		\end{itemize}
	\item The adversary has a decision: which bits, if any, should he measure?
	\item If adversary measures:
		\begin{itemize}
			\item if Alice flipped tails, gets correct value
			\item if Alice flipped heads, get correct value with probability 50\%
			\item If Alice flipped heads and Bob applied $R^{-1}$, then there is a 50\% chance Bob's bit won't match Alice's
			\item Overall, if the adversary measures, there is a 25\% chance that Bob's bit won't match alice's (chance that Alice flipped heads times chance bit collapsed to the wrong value during adversary's measurement)
		\end{itemize}
	\item The adversary might also try to apply $R^{-1}$ before measuring; in this case, he runs the risk of disrupting bits that were in the classical state to begin with
	\item Problem: the adversary needs to guess Alice�s coin flip to measure the bit without disturbing it
		\begin{itemize}
			\item if the adversary modifies more than 4 of the check bits, they will get caught with greater than 50\% probability
			\item if the adversary measures a lot, likely that integrity check will fail
			\item if the adversary doesn�t measure a lot, Alice and Bob will have a larger shared secret		
		\end{itemize}
\end{itemize}}{
% Lecture 24: 10 December 2014
\sektion{24}{Password Cracking}

\subsektion{Elementary Methods}

First, try to define and reduce the search space for a brute-force search:
\begin{itemize}
    \item all short strings
    \item combinations of dictionary words
    \item dictionary words + common modifications (special characters, exclamation point at the end)
    \item leaked passwords from past breaches
    \item dictionary words and one-character modifications
\end{itemize}

Properties of brute-force search:
\begin{itemize}
    \item Time requirement: $\sim|D|$, where $|D|$ is the length of the dictionary
    \item Space: $\sim1$
    \item Can speed up the process by pre-computing hashes of all possible passwords in search space:
    	\begin{itemize}
		\item Fill hash table: can then find $x \in D$ given $H(x)$
		\item Time to build: $\sim|D|$
		\item Space: $\sim|D|$
		\item Time to recover: $\sim1$
	\end{itemize}
    \item What we want: smaller data structure, but still fast lookup $\implies$ \emph{Rainbow Tables}
\end{itemize}

\subsektion{Rainbow Tables}
Define "reduction functions" $R_0, R_1, \cdots, R_{k-1}.$ The only requirement of these functions is that they map the output of a hash function to a string in your dictionary. The functions "reduce" since the number of hash outputs might be very large (e.g. $2^{256}$), while the dictionary is generally smaller.
\\
\\
Method: Generate chains
	\begin{itemize}
		\item Start with $a_0$, a random dictionary word. Then, compute $H(a_0)$. Next, apply $R_0$ to $H(a_0)$ to generate the next word, $a_1$. 
		\item Complete chain: $a_0 \to_H H(a_0) \to_{R_0} a_1 \to_H \cdots \to_{R_{k-2}} a_{k-1} \to_H H(a_{k-1}) \to_{R_{k-1}} a_k$
		\item Build chains with a variety of starting values $b_0, c_0$, etc...
		\item since the output of a cryptographic hash function is essentially pseudorandom, so is distribution of words that appear in a chain
		\item for each chain, remember $(a_0, a_k), (b_0, b_k)$, etc
		\item there are about $\sim \frac{|D|}{k}$ chains, so storage requirement is $\sim \frac{2|D|}{k}$
		\item k is a parameter we can adjust to control by how much we shrink storage
	\end{itemize}
	
You can use a rainbow table to recover $x \in D$ given $H(x)$ efficiently. 
	\begin{itemize}
		\item 1) Figure out which chain H(x) appears in
		\item 2) Walk that chain, we will see $x \to H(x)$ in that chain
		\item to find the chain $H(x)$ is in, guess which position in chain it is
		\item Step 1: $k^2$ time
		\item Step 2: $k$ time
	\end{itemize}}{

}\end{document}
