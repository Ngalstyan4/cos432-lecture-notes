%!TEX root = InfoSec.tex
% Lecture 9: 6 October 2014
\sektion{8}{System Security}

\sidenote{
    \textbf{The Clipper Chip}\\
    This was a chip that implemented strong symmetric encryption; however it had a \textit{law enforcement access field} (LEAF), which gave the US government a backdoor to your data. Basically, if you wanted strong crypto you sacrificed privacy in the eyes of the government.\\

    The sender uses an 80 bit sessions key, a 32 bit unit key, and a 16 bit checksum value to create a unit key. Encrypt the unit key again and you get a family key (called a leaf) that is common to all chips.
}

\subsektion{Secure system design}
Secure components
\begin{itemize}
    \item in isolation (interaction only through approved interfaces)
    \item and access control (where access control = authentication; who is asking?)
    \item plus authorizaton (does the asker have the authority?)
\end{itemize}

\subsektion{Authorization}
\begin{enumerate}
    \item Access control matrix/list (like a bouncer with a list)
    \item Capabilities (like a physical key to open a lock)
\end{enumerate}

\textbf{Access control matrix}\\
SUBJECT wants to do VERB on OBJECT\\
- Are we going to allow it?\\

Policy: a set of allowed (subject, verb, object) triplets. So we have two questions now:
\begin{enumerate}
    \item How is policy set?
    \item How is policy enforced?\\
\end{enumerate}

\subsektion{Subjects and Objects}
The subject is a process and the object is some resource (file, open network connection, window, etc). We tend to use labels to simplify policy, and set policies based on these labels.

\begin{example}
Label a process with a userid, given that there is a limited set of users.
\end{example}

But things can get complicated
\begin{example}
Alice runs a program written by Bob. The program is a text editor and the file is code for Alice's startup.\\

How to label this program?
\begin{itemize}
    \item Treat as Alice: program can steal Alice's data
    \item Treat as Bob: Alice can read Bob's files
\end{itemize}

The common approah in OS (like Linux) is to setuid, where Bob decides if the program runs as himself or the invoker.
\end{example}

\subsektion{Storing the policy information}
\textbf{Access control matrix}\\
Matrix of subjects vs. objects with allowed verbs in each cell. The downfall with these is that you get a really really big matrix with lots of empty cells; it's inefficient. 

\textbf{Profiles}\\
For each user, they have access to their row of the matrix

\textbf{Access Control List (ACL)}\\
This is the most commont approach: For each objct, (subject, verb) pair, list whether it's allowed or not

\subsektion{Who sets up the ACL?}
\begin{itemize}
    \item Centralized, top-down policy

        Pro: Might be done by someone well-trained, might be required to be top-down (eg. medical and educational records)

        Con: Inflexible, slow (for example, might have to call a help desk)
    \item Decentralized

        Pro: super flexible

        Con: mistake-prone
    \item Mixed

        Owner can choose, within limits set by a centralized authority
\end{itemize}

\textbf{Groups}\\
Logically: a set of users or groups, such that you can give access to a group.

Advantages: Makes ACLs shorter, easier to understand. The group name may describe the reason for access in the system.

\textbf{Roles}\\
If a person ``wears several hats,'' you can have a role for each ``hat''. A user can step in/out of roles.

\subsektion{Traditional Unix file access}
A file belongs to one user or one group. The ACL for each operation contains some subset of (user, group, everyone).

setuserid bit: if executed, treat as file owner if setuid == true and treat as invoker if setuid == false.

\textbf{Capabilities}\\
``The bearer may do VERB on OBJECT''\\
By definition, if you have the capability, you can do the operation.
\begin{itemize}
    \item Cryptographic

    \begin{example}
    (VERB, OBJECT, PRF(k, VERB $||$ OBJECT)) where k is known only to the system. If you know the key k, you have the capability.
    \end{example}

    Pros: totally decentralized

    Cons: if capability leaks, everyone theoretically can get capability. Revocation is hard to do.

    \item OS tracking

    OS keeps track of which capabilities you have, you name them by index
\end{itemize}

\subsektion{Authorization Logics}
Formal logics, with \textbf{primitives} for PRINCIPALS (users, groups), OBJECTS, and PERMISSIONS and \textbf{rules} for delegation.

So, a user might present a bunch of true statements (``I am part of this group, Felton says that this group has access, etc.'') and the engine identifies if the logic implies that the user has permissions to access the file.

OR the system might require the user to come up with a proof that they have access privileges, in the form of a proof with formal logic.

